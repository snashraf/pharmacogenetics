%----------------------------------------------------------------------------------------
%	PACKAGES AND OTHER DOCUMENT CONFIGURATIONS
%----------------------------------------------------------------------------------------

\documentclass{book}
\usepackage{pstricks, pst-node,pst-tree}
\usepackage{newicktree}
\usepackage{graphicx}
\usepackage{tabularx}
\usepackage{multirow}
\usepackage{xcolor}
\usepackage{hyperref}
\hypersetup{
    colorlinks,
    linkcolor=black,
    citecolor=red,
    urlcolor=cyan
}

\newcolumntype{b}{X}
\newcolumntype{s}{>{\hsize=.5\hsize}X}


%----------------------------------------------------------------------------------------
%					TITLE PAGE
%----------------------------------------------------------------------------------------

\newcommand*{\titleGM}{\begingroup % Create the command for including the title page in the document
\hbox{ % Horizontal box
\hspace*{0.1\textwidth} % Whitespace to the left of the title page
\rule{1pt}{\textheight} % Vertical line
\hspace*{0.05\textwidth} % Whitespace between the vertical line and title page text
\parbox[b]{0.5\textwidth}{ % Paragraph box which restricts text to less than the width of the page

{\noindent\Huge\bfseries Pharmacogenetic Passport}\\[2\baselineskip] % Title
{\large \textit{Pharmacogenetic guidelines and clinical annotations connected to influential DNA changes}}\\[4\baselineskip] % Tagline or further description
{\Large \textsc{ sampleName }} % Author name

\vspace{0.3\textheight} % Whitespace between the title block and the publisher
{\noindent \includegraphics{pilldna}}\\[\baselineskip] % Publisher and logo
}}
\endgroup}

%----------------------------------------------------------------------------------------
%				BLANK DOCUMENT
%----------------------------------------------------------------------------------------

\begin{document}

\pagestyle{empty} % Removes page numbers

\titleGM % This command includes the title page


% ------------- TABLE OF CONTENTS -------------------

\tableofcontents

\newpage

% ----------------  SUMMARY PAGES ----------------------

\section{Patient haplotypes}
\begin{tabularx}{\textwidth}{XXXXX}
\textbf{Gene} & \textbf{Phylogenetic method}\footnote{Method using phylogenetic trees to find closest 'relative' to patient alleles. Trees can be seen in Appendix A.} &  \textbf{Set method}\footnote{Method looking for overlap of haplotype non-reference variants and patient non-reference variants. Reference variants are filtered out of both sets prior to comparison. Top 5 hits can be seen in Appendix 2.}  \\
\hline \\
VKORC1 & H2 / H7 & H2 / *2 \\
CYPC191 & *1 / *3 & *1 / *18 \\
\end{tabularx}

\newpage

\section{Drug-gene connections}
\begin{tabularx}{\textwidth}{XXXX}
\textbf{Drug} & \textbf{Gene}  &  \textbf{Guideline} & \textbf{Annotations\footnote{ Level 1A and 1B clinical annotations meet the highest levels of criteria and are manually curated by PharmGKB. Level 1A annotations contain a variant-drug combination in a CPIC or medical society endorsed PGx guideline, or, implemented at a PGRN site, or, in another major health system. Level 1B annotations contain a variant-drug combination where the preponderance of evidence shows an association. The association must be replicated in more than one cohort with significant p-values, and, preferably with a strong effect size. Lower levels (3-4) are less significant and may only be based on a single study or case report, which may be performed in vitro.(PHARMGKB) }}  \\
\hline \\
Warfarin & VKORC1 &  \href{http://www.wikibooks.org}{Yes} & \textit{Level 1-2}: 15 \newline \textit{Level 3-4}: 13 \\
Warfarin & CYPC191 &  Yes & \textit{Level 1-2}: 15 \newline \textit{Level 3-4}: 13 \\
\end{tabularx}

\newpage

% ----------------  Guidelines ----------------------

\section{Haplotype Guidelines}

\subsection{Acenocoumarol}

\subsubsection{VKORC1}

\begin{center}
Patient haplotype
\textbf{ H1/H2 } | \textbf{ H1/*2 } \newline\newline


\scriptsize
\begin{tabularx}{\textwidth}{ssbb}
\textbf{Genotype} & \textbf{Therapeutic Dose Recommendation} & \textbf{ Level of Evidence} & \textbf{Clinical Relevance} \\ \\
\hline \\ \vspace{1pt}
VKORC1 [variant:rs9934438] AG & None & Published controlled studies of good quality* relating to phenotyped and/or genotyped patients or healthy volunteers, and having relevant pharmacokinetic or clinical endpoints. & Minor clinical effect (S): QTc prolongation (;450 ms ;470 ms); INR increase; 4.5 Kinetic effect (S). \\
\\ \vspace{1pt}
VKORC1 [variant:rs9934438] AA & Check INR more frequently. & Published controlled studies of good quality* relating to phenotyped and/or genotyped patients or healthy volunteers, and having relevant pharmacokinetic or clinical endpoints. & Minor clinical effect (S): QTc prolongation (;450 ms , ;470 ms); INR increase ; 4.5 Kinetic effect (S). \\
\end{tabularx}

\end{center}
\newpage
\normalsize

% ----------------  High level Clinical Variations ----------------------

\section{Clinical Annotations}

\subsection{Acenocoumarol}
\subsubsection{VKORC1}
\textbf{\colorbox{cyan} {Class 2A}} \textbf{ rs9934438 } \textit{ GA }
Patients with the AG genotype may have decreased dose of acenocoumarol or phenprocoumon as compared to patients with genotype GG. Other genetic and clinical factors may also influence the dose of acenocoumarol or phenprocoumon. 
\subsubsection{PRSS53}
\textbf{\colorbox{cyan} {Class 2A}} \textbf{ rs9934438 } \textit{ GA }
Patients with the AG genotype may have decreased dose of acenocoumarol or phenprocoumon as compared to patients with genotype GG. Other genetic and clinical factors may also influence the dose of acenocoumarol or phenprocoumon. 

\newpage

% ----------------  APPENDICES ----------------------

\section{Appendix 1: Haplotype trees}
\begin{center}
\textbf{PA125}
\begin{newicktree}
\nobranchlengths
\nonodemarkers
\righttree \setunitlength{0.5cm}
\drawtree{
(
(
PA165958681:0
,
PatientAllele1:0
):0
,
PatientAllele2:0
);}\end{newicktree} 
\end{center}

\section{Appendix 2: Top 3 haplotypes}

\begin{tabularx}{\textwidth}{ c c c }
test & \textbf{Allele 1} & \textbf{Allele 2} \\
\multirow{3}{200pt}{CYP2D6} &
\**1 & *1 \\
\**2 & *2 \\
\**3 & *3 \\
\end{tabularx}

\section{Appendix 3: Low level annotations}
Nyanyanya


% ------------------
\end{document}