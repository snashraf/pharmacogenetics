\documentclass{resume} % Use the custom resume.cls style

\usepackage[left=0.75in,top=0.6in,right=0.75in,bottom=0.6in]{geometry} % Document margins
\usepackage{xcolor}

\name{ STE0097 } % Your name
\address{Generated on \today} % Your address

\begin{document}

\begin{rSection}{ Ace Inhibitors, Plain }
\item[]

\begin{rSubsection}{ ADRB2 }{ adrenoceptor beta 2, surface }{}{}
\item[]

\item[] ---------------------------------------------------- Clinical Annotations -----------------------------------------------------\newline
\item \textbf{\colorbox{teal} {Class 3}} \textbf{ rs1042713 } \textit{ GG }
\item[] Patients with the GG genotype and heart failure may have increased emergency department visits and hospital utilization when treated with cardiovascular drugs as compared to patients with the AA or AG genotype. Other genetic and clinical factors may also influence efficacy of cardiovascular drugs.
\end{rSubsection}

\end{rSection}\begin{rSection}{ Analgesics }
\item[]

\begin{rSubsection}{ COMT }{ catechol-O-methyltransferase }{}{}
\item[]

\item[] ---------------------------------------------------- Clinical Annotations -----------------------------------------------------\newline
\item \textbf{\colorbox{teal} {Class 3}} \textbf{ rs4680 } \textit{ GA }
\item[] Patients with the AG genotype with substance withdrawal syndrome may have an increased likelihood of headache when discontinuing the use of analgesics (such as opioids, NSAIDs, triptans, ergot) as compared to patients with the AA genotype. Other clinical and genetic factors may also influence likelihood of headache in patients with withdrawal syndrome who discontinue the use of analgesics.
\end{rSubsection}

\end{rSection}\begin{rSection}{ Antiinflammatory agents, non-steroids }
\item[]

\begin{rSubsection}{ COMT }{ catechol-O-methyltransferase }{}{}
\item[]

\item[] ---------------------------------------------------- Clinical Annotations -----------------------------------------------------\newline
\item \textbf{\colorbox{teal} {Class 3}} \textbf{ rs4680 } \textit{ GA }
\item[] Patients with the AG genotype with substance withdrawal syndrome may have an increased likelihood of headache when discontinuing the use of analgesics (such as opioids, NSAIDs, triptans, ergot) as compared to patients with the AA genotype. Other clinical and genetic factors may also influence likelihood of headache in patients with withdrawal syndrome who discontinue the use of analgesics.
\end{rSubsection}

\end{rSection}\begin{rSection}{ Antivirals for treatment of HIV infections, combinations }
\item[]

\begin{rSubsection}{ ABCB1 }{ ATP-binding cassette, sub-family B (MDR/TAP), member 1 }{}{}
\item[]

\item[] ---------------------------------------------------- Clinical Annotations -----------------------------------------------------\newline
\item \textbf{\colorbox{teal} {Class 3}} \textbf{ rs1045642 } \textit{ GG }
\item[] Patients with the GG genotype who are co-infected with HIV and tuberculosis (TB) may have a decreased risk for hepatotoxicity when treated with anti-tubercular and antiretroviral drugs as compared to patients with the AA genotype. Other genetic and clinical factors may also influence risk of hepatotoxicity.
\end{rSubsection}

\end{rSection}\begin{rSection}{ Beta Blocking Agents }
\item[]

\begin{rSubsection}{ ADRB2 }{ adrenoceptor beta 2, surface }{}{}
\item[]

\item[] ---------------------------------------------------- Clinical Annotations -----------------------------------------------------\newline
\item \textbf{\colorbox{teal} {Class 3}} \textbf{ rs1042713 } \textit{ GG }
\item[] Patients with the GG genotype and heart failure may have increased emergency department visits and hospital utilization when treated with cardiovascular drugs as compared to patients with the AA or AG genotype. Other genetic and clinical factors may also influence efficacy of cardiovascular drugs.
\end{rSubsection}

\end{rSection}\begin{rSection}{ Dabigatran }
\item[]

\begin{rSubsection}{ ABCB1 }{ ATP-binding cassette, sub-family B (MDR/TAP), member 1 }{}{}
\item[]

\item[] ---------------------------------------------------- Clinical Annotations -----------------------------------------------------\newline
\item \textbf{\colorbox{green} {Class 4}} \textbf{ rs1045642 } \textit{ GG }
\item[] People with the GG genotype may have decreased exposure to dabigatran compared to patients with the AA and AG genotypes, when also assessed with the rs2032582 allele. Other clinical and genetic factors may affect exposure to dabigatran.\item \textbf{\colorbox{green} {Class 4}} \textbf{ rs2032582 } \textit{ CC }
\item[] People with the CC genotype may have decreased exposure to dabigatran compared to patients with a variant at this position, including genotypes AA, AC, CT, and TT, when assessed in conjunction with a variant at position rs1045642. Other clinical and genetic factors may affect exposure to dabigatran. 
\end{rSubsection}

\end{rSection}\begin{rSection}{ Drugs For Treatment Of Tuberculosis }
\item[]

\begin{rSubsection}{ ABCB1 }{ ATP-binding cassette, sub-family B (MDR/TAP), member 1 }{}{}
\item[]

\item[] ---------------------------------------------------- Clinical Annotations -----------------------------------------------------\newline
\item \textbf{\colorbox{teal} {Class 3}} \textbf{ rs1045642 } \textit{ GG }
\item[] Patients with the GG genotype and tuberculosis (TB) may have a decreased risk for hepatotoxicity when treated with anti-TB drugs as compared to patients with the AA genotype. Other genetic and clinical factors may also influence hepatotoxicity.\item \textbf{\colorbox{teal} {Class 3}} \textbf{ rs1045642 } \textit{ GG }
\item[] Patients with the GG genotype who are co-infected with HIV and tuberculosis (TB) may have a decreased risk for hepatotoxicity when treated with anti-tubercular and antiretroviral drugs as compared to patients with the AA genotype. Other genetic and clinical factors may also influence risk of hepatotoxicity.
\end{rSubsection}

\end{rSection}\begin{rSection}{ Opium alkaloids and derivatives }
\item[]

\begin{rSubsection}{ CHRNA3 }{ cholinergic receptor, nicotinic, alpha 3 (neuronal) }{}{}
\item[]

\item[] ---------------------------------------------------- Clinical Annotations -----------------------------------------------------\newline
\item \textbf{\colorbox{teal} {Class 3}} \textbf{ rs16969968 } \textit{ GG }
\item[] Patients with the GG genotype who are in chronic pain and receive opioid medications for treatment may be at decreased risk for addiction as compared to patients with the AA genotype. Other genetic and clinical factors may also influence risk of opiate addiction.
\end{rSubsection}

\end{rSection}\begin{rSection}{ Platinum compounds }
\item[]

\begin{rSubsection}{ ABCB1 }{ ATP-binding cassette, sub-family B (MDR/TAP), member 1 }{}{}
\item[]

\item[] ---------------------------------------------------- Clinical Annotations -----------------------------------------------------\newline
\item \textbf{\colorbox{teal} {Class 3}} \textbf{ rs1045642 } \textit{ GG }
\item[] Patients with the GG genotype and non-small-cell lung cancer may have a better response to platinum-based chemotherapy as compared to patients with the AA or AG genotype. This was only seen in those of Asian ethnicity. Other genetic and clinical factors may also influence response to platinum-based chemotherapy.\item \textbf{\colorbox{teal} {Class 3}} \textbf{ rs1128503 } \textit{ GG }
\item[] Patients with the GG genotype and non-small cell lung cancer may have reduced risk of toxicities when treated with platinum-based chemotherapy compared to patients with the AA genotype. Other clinical and genetic factors may affect risk of toxicities in response to platinum-based chemotherapies.
\end{rSubsection}

\end{rSection}\begin{rSection}{ Pyrimidine analogues }
\item[]

\begin{rSubsection}{ DPYD }{ dihydropyrimidine dehydrogenase }{}{}
\item[]

\item[] ---------------------------------------------------- Clinical Annotations -----------------------------------------------------\newline
\item \textbf{\colorbox{red} {Class 1A}} \textbf{ rs55886062 } \textit{ AA }
\item[] Patients with the AA genotype (DPYD *1/*1) and cancer who are treated with fluoropyrimidine-based chemotherapy may have a decreased, but not absent, risk for drug toxicity as compared to patients with the AC or CC genotype (DPYD *1/*13 or *13/*13). Fluoropyrimidines are often used in combination chemotherapy such as FOLFOX (fluorouracil, leucovorin and oxaliplatin), FOLFIRI (fluorouracil, leucovorin and irinotecan) or FEC (fluorouracil, epirubicin and cyclophosphamide) or with other drugs such as bevacizumab, cetuximab, raltitrexed. The combination and delivery of the drug may influence risk for toxicity. Other genetic and clinical factors may also influence response to fluoropyrimidine-based chemotherapy.\item \textbf{\colorbox{red} {Class 1A}} \textbf{ rs3918290 } \textit{ CC }
\item[] Patients with the CC genotype (DPYD *1/*1) and cancer who are treated with fluoropyrimidine-based chemotherapy may have 1) increased clearance of fluoropyrimidine drugs and 2) decreased, but not non-existent, risk for drug toxicity as compared to patients with the CT or TT genotype (DPYD *1/*2A or *2A/*2A). Fluoropyrimidines are often used in combination chemotherapy such as FOLFOX (fluorouracil, leucovorin and oxaliplatin), FOLFIRI (fluorouracil,  leucovorin and irinotecan) or FEC (fluorouracil, epirubicin and cyclophosphamide) or with other drugs such as bevacizumab, cetuximab, raltitrexed. The combination and delivery of the drug may influence risk for toxicity. Other genetic and clinical factors may also influence response to fluoropyrimidine based chemotherapy.\item \textbf{\colorbox{red} {Class 1A}} \textbf{ rs67376798 } \textit{ TT }
\item[] Patients with the TT genotype and cancer who are treated with fluoropyrimidine-based chemotherapy may have 1) increased clearance of the drug and 2) decreased, but not absent, risk and reduced severity of drug toxicity as compared to patients with the AT genotype. Fluoropyrimidines are often used in combination chemotherapy such as FOLFOX (fluorouracil, leucovorin and oxaliplatin), FOLFIRI (fluorouracil, leucovorin and irinotecan) or FEC (fluorouracil, epirubicin and cyclophosphamide) or with other drugs such as bevacizumab, cetuximab, raltitrexed. The combination and delivery of the drug may influence risk for toxicity. Other genetic and clinical factors may also influence response to fluoropyrimidine-based chemotherapy.\item \textbf{\colorbox{teal} {Class 3}} \textbf{ rs1801159 } \textit{ TT }
\item[] Patients with the TT genotype (DPYD *1/*1) and cancer who are treated with fluoropyrimidine-based chemotherapy may have 1) a decreased likelihood of nausea, vomiting, and leukopenia, 2) increased response and 3) increased clearance of fluorouracil as compared to patients with the CT or CC genotype (DPYD *1/*5 or *5/*5). However, other studies find no associations or contradictory associations with fluoropyrimidine-induced drug toxicity or response. Other genetic and clinical factors may also influence response to fluoropyrimidine-based chemotherapy.
\end{rSubsection}

\end{rSection}\begin{rSection}{ acenocoumarol }
\item[]

\begin{rSubsection}{ CYP2C9 }{ cytochrome P450, family 2, subfamily C, polypeptide 9 }{}{}
\item[]

\item[] ---------------------------------------------------- Clinical Annotations -----------------------------------------------------\newline
\item \textbf{\colorbox{cyan} {Class 2A}} \textbf{ rs1057910 } \textit{ AC }
\item[] Patients with the AC genotype may require decreased dose of acenocoumarol or closer INR monitoring as compared to patients with the AA genotype. Other genetic and clinical factors may also influence acenocoumarol dose.\item \textbf{\colorbox{teal} {Class 3}} \textbf{ rs1799853 } \textit{ CC }
\item[] Patients with the CC genotype who are taking acenocoumarol may have a decreased risk of a gastrointestinal hemorrhage as compared to patients with the CT or TT genotype. Other genetic and clinical factors may also influence risk of gastrointestinal hemorrhage.
\end{rSubsection}

\end{rSection}\begin{rSection}{ acetaminophen }
\item[]

\begin{rSubsection}{ UGT1A }{ UDP glucuronosyltransferase 1 family, polypeptide A complex locus }{}{}
\item[]

\item[] ---------------------------------------------------- Clinical Annotations -----------------------------------------------------\newline
\item \textbf{\colorbox{teal} {Class 3}} \textbf{ rs8330 } \textit{ GC }
\item[] Patients with the CG genotype may have a decreased risk of liver failure due to unintentional acetaminophen overdose as compared to patients with the CC genotype. Other genetic and clinical factors may also influence risk of liver failure due to unintentional acetaminophen overdose.\item \textbf{\colorbox{teal} {Class 3}} \textbf{ rs1042640 } \textit{ GC }
\item[] Patients with the CG genotype may have a decreased risk of liver failure due to unintentional acetaminophen overdose as compared to patients with the CC genotype. Other genetic and clinical factors may also influence risk of liver failure due to unintentional acetaminophen overdose.\item \textbf{\colorbox{teal} {Class 3}} \textbf{ rs10929303 } \textit{ TC }
\item[] Patients with the CT genotype may have a decreased risk of liver failure due to unintentional acetaminophen overdose as compared to patients with the CC genotype. Other genetic and clinical factors may also influence risk of liver failure due to unintentional acetaminophen overdose.
\end{rSubsection}

\end{rSection}\begin{rSection}{ amlodipine }
\item[]

\begin{rSubsection}{ CYP3A4 }{ cytochrome P450, family 3, subfamily A, polypeptide 4 }{}{}
\item[]

\item[] ---------------------------------------------------- Clinical Annotations -----------------------------------------------------\newline
\item \textbf{\colorbox{teal} {Class 3}} \textbf{ rs2740574 } \textit{ TT }
\item[] Women with the TT genotype and hypertension may have an increased likelihood of reaching a target mean arterial pressure of <= 107 mm Hg when treated with amlodipine as compared to women with the CC genotype. No significant associations were seen when considering a target mean arterial pressure of <= 92 mm Hg, or when considering men or men and women together. Other genetic and clinical factors may also influence response to amlodipine.
\end{rSubsection}

\end{rSection}\begin{rSection}{ anastrozole }
\item[]

\begin{rSubsection}{ UGT1A10 }{ UDP glucuronosyltransferase 1 family, polypeptide A10 }{}{}
\item[]

\item[] ---------------------------------------------------- Clinical Annotations -----------------------------------------------------\newline
\item \textbf{\colorbox{green} {Class 4}} \textbf{ rs3732219 } \textit{ CC }
\item[] Patients with the CC genotype may have increased glucuronidation of anastrozole as compared to patients with the CT or TT genotype, as determined by in vitro assays. Glucuronidation allows for the elimination of xenobiotics like anastrozole. Other genetic and clinical factors may also influence glucuronidation of anastrozole. \item \textbf{\colorbox{green} {Class 4}} \textbf{ rs3732218 } \textit{ GG }
\item[] Patients with the GG genotype may have increased glucuronidation of anastrozole as compared to patients with the AA or AG genotype, as determined by in vitro assays. Glucuronidation allows for the elimination of xenobiotics like anastrozole. Other genetic and clinical factors may also influence glucuronidation of anastrozole.
\end{rSubsection}

\end{rSection}\begin{rSection}{ anthracyclines and related substances }
\item[]

\begin{rSubsection}{ ABCB1 }{ ATP-binding cassette, sub-family B (MDR/TAP), member 1 }{}{}
\item[]

\item[] ---------------------------------------------------- Clinical Annotations -----------------------------------------------------\newline
\item \textbf{\colorbox{teal} {Class 3}} \textbf{ rs1045642 } \textit{ GG }
\item[] Patients with the GG genotype may have 1) decreased exposure to doxorubicin metabolites and 2) decreased response to anthracycline regimens as compared to patients with the AA genotype, however the evidence is highly contradictory. Other genetic and clinical factors may also influence response to anthracycline regimens.
\end{rSubsection}

\end{rSection}\begin{rSection}{ antiepileptics }
\item[]

\begin{rSubsection}{ ABCB1 }{ ATP-binding cassette, sub-family B (MDR/TAP), member 1 }{}{}
\item[]

\item[] ---------------------------------------------------- Clinical Annotations -----------------------------------------------------\newline
\item \textbf{\colorbox{teal} {Class 3}} \textbf{ rs1128503 } \textit{ GG }
\item[] Patients with the GG genotype and specifically localization-related epilepsy syndrome may have a decreased risk for resistance to antiepileptic treatment as compared to patients with the AA genotype. However, all other studies of people with epilepsy have found no association between this variant and antiepileptic resistance. Other genetic and clinical factors may also influence resistance to antiepileptics. 
\end{rSubsection}

\end{rSection}\begin{rSection}{ antipsychotics }
\item[]

\begin{rSubsection}{ COMT }{ catechol-O-methyltransferase }{}{}
\item[]

\item[] ---------------------------------------------------- Clinical Annotations -----------------------------------------------------\newline
\item \textbf{\colorbox{teal} {Class 3}} \textbf{ rs4680 } \textit{ GA }
\item[] Patients with the AG genotype may have increased blood pressure when treated with antipsychotics as compared to patients with the GG genotype. Other genetic and clinical factors may also influence blood pressure in patients receiving antipsychotics.\item \textbf{\colorbox{teal} {Class 3}} \textbf{ rs4680 } \textit{ GA }
\item[] Patients with the AG genotype may have increased fasting glucose levels when treated with antipsychotics as compared to patients with the GG genotype. Other genetic and clinical factors may also influence fasting glucose in patients taking antipsychotics.
\end{rSubsection}

\end{rSection}\begin{rSection}{ atazanavir }
\item[]

\begin{rSubsection}{ UGT1A }{ UDP glucuronosyltransferase 1 family, polypeptide A complex locus }{}{}
\item[]

\item[] ---------------------------------------------------- Clinical Annotations -----------------------------------------------------\newline
\item \textbf{\colorbox{teal} {Class 3}} \textbf{ rs10929303 } \textit{ TC }
\item[] Patients with the CT genotype and HIV may have a decreased risk of nephrolithiasis when treated with atazanavir and ritonavir as compared to patients with the TT genotype and an increased risk of nephrolithiasis as compared to people with the CC genotype. Other genetic and clinical factors may also affect risk of nephrolithiasis in patients with HIV who are taking atazanavir and ritonavir. \item \textbf{\colorbox{teal} {Class 3}} \textbf{ rs1042640 } \textit{ GC }
\item[] Patients with the CG genotype and HIV may have a decreased risk of nephrolithiasis when treated with atazanavir and ritonavir as compared to patients with the GG genotype and an increased risk of nephrolithiasis as compared to patients with the CC genotype. Other genetic and clinical factors may also affect risk of nephrolithiasis in people with HIV who are taking atazanavir and ritonavir.\item \textbf{\colorbox{teal} {Class 3}} \textbf{ rs8330 } \textit{ GC }
\item[] Patients with the CG genotype and HIV may have a decreased risk of nephrolithiasis when treated with atazanavir and ritonavir as compared to patients with the GG genotype and an increased risk of nephrolithiasis as compared to people with the CC genotype. Other genetic and clinical factors may also affect risk of nephrolithiasis in patients with HIV who are taking atazanavir and ritonavir.
\end{rSubsection}

\end{rSection}\begin{rSection}{ atenolol }
\item[]

\begin{rSubsection}{ ADRB2 }{ adrenoceptor beta 2, surface }{}{}
\item[]

\item[] ---------------------------------------------------- Clinical Annotations -----------------------------------------------------\newline
\item \textbf{\colorbox{teal} {Class 3}} \textbf{ rs1042714 } \textit{ GG }
\item[] Patients with the GG genotype and hypertension may have an increased risk of developing hypertriglyceridemia when treated with atenolol or metoprolol as compared to patients with the CC or CG genotype. Other genetic and clinical factors may also influence risk of hypertriglyceridemia. 
\end{rSubsection}

\end{rSection}\begin{rSection}{ atorvastatin }
\item[]

\begin{rSubsection}{ ABCC2 }{ ATP-binding cassette, sub-family C (CFTR/MRP), member 2 }{}{}
\item[]

\item[] ---------------------------------------------------- Clinical Annotations -----------------------------------------------------\newline
\item \textbf{\colorbox{teal} {Class 3}} \textbf{ rs717620 } \textit{ CT }
\item[] Patients with the CT genotype may have decreased dose of simvastatin and atorvastatin as compared to patients with genotype CC. Other genetic and clinical factors may also influence the dose of simvastatin.
\end{rSubsection}

\end{rSection}\begin{rSection}{ benazepril }
\item[]

\begin{rSubsection}{ ADRB2 }{ adrenoceptor beta 2, surface }{}{}
\item[]

\item[] ---------------------------------------------------- Clinical Annotations -----------------------------------------------------\newline
\item \textbf{\colorbox{teal} {Class 3}} \textbf{ rs1042713 } \textit{ GG }
\item[] Patients with the GG genotype and hypertension may have a greater decrease in diastolic blood pressure when treated with benazepril as compared to patients with the AA genotype. No significant results have been seen for systolic blood pressure. Additionally, the same study reported no significant differences in systolic or diastolic blood pressure between genotypes in a different cohort. Other genetic and clinical factors may also influence change in diastolic or systolic blood pressure.
\end{rSubsection}

\end{rSection}\begin{rSection}{ bevacizumab }
\item[]

\begin{rSubsection}{ VEGFA }{ vascular endothelial growth factor A }{}{}
\item[]

\item[] ---------------------------------------------------- Clinical Annotations -----------------------------------------------------\newline
\item \textbf{\colorbox{teal} {Class 3}} \textbf{ rs2010963 } \textit{ CG }
\item[] Patients with the CG genotype and choroidal neovascularization may have a better response to anti-VEGF treatment, as compared to patients with the CC genotype. Other genetic and clinical factors may also influence response to anti-VEGF treatment. \item \textbf{\colorbox{teal} {Class 3}} \textbf{ rs699947 } \textit{ AC }
\item[] Patients with colorectal cancer and the AC genotype may have a reduced response to bevacizumab, capecitabine, fluorouracil, irinotecan, leucovorin, or oxaliplatin as compared to patients with the CC genotype. Other clinical and genetic factors may also affect response to chemotherapy in people with colorectal cancer.
\end{rSubsection}

\end{rSection}\begin{rSection}{ bupropion }
\item[]

\begin{rSubsection}{ CYP2B6 }{ cytochrome P450, family 2, subfamily B, polypeptide 6 }{}{}
\item[]

\item[] ---------------------------------------------------- Clinical Annotations -----------------------------------------------------\newline
\item \textbf{\colorbox{teal} {Class 3}} \textbf{ rs3211371 } \textit{ CC }
\item[] Patients with the CC genotype who are smokers may have a lower chance of smoking cessation when treated with bupropion as compared to patients with the CT or TT genotype, although this is contradicted in one study. Other genetic and clinical factors may also influence likelihood of smoking cessation.\item \textbf{\colorbox{teal} {Class 3}} \textbf{ rs2279343 } \textit{ AA }
\item[] Individuals with tobacco use disorder and the AA genotype may have an improved response to bupropion as compared to individuals with the AG and GG genotypes. Other clinical and genetic factors may also affect response to bupropion in individuals with tobacco use disorder. 
\end{rSubsection}

\end{rSection}\begin{rSection}{ busulfan }
\item[]

\begin{rSubsection}{ CYP2C19 }{ cytochrome P450, family 2, subfamily C, polypeptide 19 }{}{}
\item[]

\item[] ---------------------------------------------------- Clinical Annotations -----------------------------------------------------\newline
\item \textbf{\colorbox{teal} {Class 3}} \textbf{ rs12248560 } \textit{ CC }
\item[] Patients with the CC genotype (CYP2C19 *1/*1) undergoing transplantation may have decreased metabolism of busulfan as compared to patients with the CT (*1/*17) or TT (*17/*17) genotype. However, some contradictory evidence exists for this association. Other genetic and clinical factors may also influence metabolism of busulfan.
\end{rSubsection}

\end{rSection}\begin{rSection}{ capecitabine }
\item[]

\begin{rSubsection}{ ABCB1 }{ ATP-binding cassette, sub-family B (MDR/TAP), member 1 }{}{}
\item[]

\item[] ---------------------------------------------------- Clinical Annotations -----------------------------------------------------\newline
\item \textbf{\colorbox{teal} {Class 3}} \textbf{ rs1045642 } \textit{ GG }
\item[] Patients with GG genotype may have increased risk of hand-foot syndrome when treated with capecitabine in people with Colorectal Neoplasms as compared to patients with genotype AA. Genotypes AG + GG are not associated with decreased clinical outcome when treated with capecitabine, cisplatin, docetaxel, epirubicin and gemcitabine in people with Pancreatic Neoplasms as compared to genotype AA. Other genetic and clinical factors may influence the response to capecitabine.\item \textbf{\colorbox{teal} {Class 3}} \textbf{ rs2032582 } \textit{ CC }
\item[] Patients with genotype CC may have increased risk of hand-foot syndrome when treated with capecitabine in people with Colorectal Neoplasms as compared to patients with genotype AA. Other genetic and clinical factors may also influence the response to capecitabine.\item \textbf{\colorbox{teal} {Class 3}} \textbf{ rs1128503 } \textit{ GG }
\item[] Patients with the GG genotype and colorectal cancer may have an increased risk of neutropenia or hand-foot syndrome when treated with capecitabine as compared to patients with the AA genotype. Other genetic and clinical factors may also influence risk of neutropenia or hand-foot syndrome.
\end{rSubsection}

\end{rSection}\begin{rSection}{ carbamazepine }
\item[]

\begin{rSubsection}{ SCN1A }{ sodium channel, voltage-gated, type I, alpha subunit }{}{}
\item[]

\item[] ---------------------------------------------------- Clinical Annotations -----------------------------------------------------\newline
\item \textbf{\colorbox{blue} {Class 2B}} \textbf{ rs3812718 } \textit{ CT }
\item[] Patients with the CT genotype who are treated with carbamazepine may require a higher dose as compared to patients with the CC genotype. Other genetic and clinical factors may also influence dose of carbamazepine.\item \textbf{\colorbox{blue} {Class 2B}} \textbf{ rs3812718 } \textit{ CT }
\item[] Patients with the CT genotype and epilepsy may be less likely to be resistant to antiepileptic treatment, particularly carbamazepine, as compared to patients with the TT genotype. Other genetic and clinical factors may also influence resistance to antiepileptic drugs.\item \textbf{\colorbox{teal} {Class 3}} \textbf{ rs3812718 } \textit{ CT }
\item[] Patients with epilepsy and the CT genotype may have decreased metabolism of carbamazepine, resulting in increased exposure as compared to patients with the TT genotype. 
\end{rSubsection}

\end{rSection}\begin{rSection}{ carvedilol }
\item[]

\begin{rSubsection}{ UGT1A10 }{ UDP glucuronosyltransferase 1 family, polypeptide A10 }{}{}
\item[]

\item[] ---------------------------------------------------- Clinical Annotations -----------------------------------------------------\newline
\item \textbf{\colorbox{teal} {Class 3}} \textbf{ rs4148323 } \textit{ GG }
\item[] Patients with the GG (i.e. UGT1A1 *1/*1) genotype and angina or heart failure may have increased glucuronidation of carvedilol as compared to patients with the AA (*6/*6) genotype. UGT1A1 is responsible for the glucuronidation of target substrates, rendering them water soluble and allowing for their biliary or renal elimination. Other genetic and clinical factors may also influence metabolism of carvedilol.
\end{rSubsection}

\end{rSection}\begin{rSection}{ catecholamines }
\item[]

\begin{rSubsection}{ ADRB1 }{ adrenoceptor beta 1 }{}{}
\item[]

\item[] ---------------------------------------------------- Clinical Annotations -----------------------------------------------------\newline
\item \textbf{\colorbox{teal} {Class 3}} \textbf{ rs1801253 } \textit{ GG }
\item[] Patients with the GG genotype and coronary artery disease may require an increased dose of catecholamines as compared to patients with the CC or CG genotype. Other genetic and clinical factors may also influence required dose of catecholamines.
\end{rSubsection}

\end{rSection}\begin{rSection}{ celecoxib }
\item[]

\begin{rSubsection}{ CYP2C9 }{ cytochrome P450, family 2, subfamily C, polypeptide 9 }{}{}
\item[]

\item[] ---------------------------------------------------- Clinical Annotations -----------------------------------------------------\newline
\item \textbf{\colorbox{cyan} {Class 2A}} \textbf{ rs1057910 } \textit{ AC }
\item[] Patients with the AC (CYP2C9 *1/*3) genotype may have reduced metabolism of celecoxib as compared to patients with the AA (*1/*1) genotype, and increased metabolism as compared to patients with the CC (*3/*3) genotype. Other genetic and clinical factors may also influence metabolism of celecoxib. 
\end{rSubsection}

\end{rSection}\begin{rSection}{ clopidogrel }
\item[]

\begin{rSubsection}{ ABCB1 }{ ATP-binding cassette, sub-family B (MDR/TAP), member 1 }{}{}
\item[]

\item[] ---------------------------------------------------- Clinical Annotations -----------------------------------------------------\newline
\item \textbf{\colorbox{teal} {Class 3}} \textbf{ rs1045642 } \textit{ GG }
\item[] People with GG  genotype may have decreased, but not absent, risk of major adverse cardiovascular events (MACE such as cardiovascular death, myocardial infarction, or stroke) when treated with clopidogrel in people with acute coronary syndrome or myocardial Infarction as compared to people with genotypes AA. Contradictory findings have been reported in the literature. Other genetic and clinical factors may also impact the response to clopidogrel.
\end{rSubsection}

\end{rSection}\begin{rSection}{ clozapine }
\item[]

\begin{rSubsection}{ COMT }{ catechol-O-methyltransferase }{}{}
\item[]

\item[] ---------------------------------------------------- Clinical Annotations -----------------------------------------------------\newline
\item \textbf{\colorbox{teal} {Class 3}} \textbf{ rs4680 } \textit{ GA }
\item[] Patients with the AG genotype and schizophrenia may have a poorer response when treated with clozapine as compared to patients with the GG genotype. Other genetic and clinical factors may also influence response to clozapine.
\end{rSubsection}

\end{rSection}\begin{rSection}{ codeine }
\item[]

\begin{rSubsection}{ ABCB1 }{ ATP-binding cassette, sub-family B (MDR/TAP), member 1 }{}{}
\item[]

\item[] ---------------------------------------------------- Clinical Annotations -----------------------------------------------------\newline
\item \textbf{\colorbox{teal} {Class 3}} \textbf{ rs1128503 } \textit{ GG }
\item[] Breast-feeding infants whose mothers have the GG genotype and are taking codeine may be at decreased risk for CNS depression as compared to those whose mothers have the AA genotype. Other genetic and clinical factors may also influence the risk of CNS depression in breast-feeding infants. 
\end{rSubsection}

\end{rSection}\begin{rSection}{ cotinine }
\item[]

\begin{rSubsection}{ CHRNA3 }{ cholinergic receptor, nicotinic, alpha 3 (neuronal) }{}{}
\item[]

\item[] ---------------------------------------------------- Clinical Annotations -----------------------------------------------------\newline
\item \textbf{\colorbox{teal} {Class 3}} \textbf{ rs16969968 } \textit{ GG }
\item[] Individuals with Tobacco Use Disorder and the GG genotype may have decreased concentrations of cotinine, a metabolite of nicotine, as compared to individuals with the AG or AA genotype. Other clinical and genetic factors may also contribute to cotinine concentrations in individuals with Tobacco Use Disorder.
\end{rSubsection}

\end{rSection}\begin{rSection}{ coumarin }
\item[]

\begin{rSubsection}{ CYP2A6 }{ cytochrome P450, family 2, subfamily A, polypeptide 6 }{}{}
\item[]

\item[] ---------------------------------------------------- Clinical Annotations -----------------------------------------------------\newline
\item \textbf{\colorbox{green} {Class 4}} \textbf{ rs1801272 } \textit{ AT }
\item[] Patients with the AT genotype may have increased 7-hydroxylation of coumarin compared to patients with the TT genotype. Other genetic and clinical factors may also influence metabolism of coumarin.
\end{rSubsection}

\end{rSection}\begin{rSection}{ cyclophosphamide }
\item[]

\begin{rSubsection}{ CYP3A4 }{ cytochrome P450, family 3, subfamily A, polypeptide 4 }{}{}
\item[]

\item[] ---------------------------------------------------- Clinical Annotations -----------------------------------------------------\newline
\item \textbf{\colorbox{teal} {Class 3}} \textbf{ rs2740574 } \textit{ TT }
\item[] Premenopausal patients with the TT genotype and breast cancer who are treated with cyclophosphamide may have a shorter period of time before chemotherapy-induced ovarian failure compared to patients with the CC or CT genotype. Other genetic and clinical factors may also influence time to chemotherapy-induced ovarian failure.
\end{rSubsection}

\end{rSection}\begin{rSection}{ cyclosporine }
\item[]

\begin{rSubsection}{ ABCB1 }{ ATP-binding cassette, sub-family B (MDR/TAP), member 1 }{}{}
\item[]

\item[] ---------------------------------------------------- Clinical Annotations -----------------------------------------------------\newline
\item \textbf{\colorbox{teal} {Class 3}} \textbf{ rs1128503 } \textit{ GG }
\item[] Patients with the GG genotype and myasthenia gravis or organ transplantation may have increased clearance of cyclosporine and therefore may require an increased dose of cyclosporine, compared to patients with the AA genotype. Patients with the GG genotype may also have a decreased risk of infection as compared to those with the AA or AG genotype. Other genetic and clinical factors may also influence clearance and dose of cyclosporine.\item \textbf{\colorbox{teal} {Class 3}} \textbf{ rs2032582 } \textit{ CC }
\item[] Patients with the CC genotype may have lower blood trough concentrations of cyclosporine compared to patients with the AA genotype, and may require dose adjustments. Other genetic and clinical factors may also influence cyclosporine blood concentrations.\item \textbf{\colorbox{teal} {Class 3}} \textbf{ rs1045642 } \textit{ GG }
\item[] Patients with genotype GG may have decreased intracellular and blood concentrations of cyclosporine in people with Transplantation as compared to patients with genotype AA or AG. However, contradictory findings have been reported. Other genetic and clinical factors may also influence the concentration of cyclosporine.\item \textbf{\colorbox{teal} {Class 3}} \textbf{ rs2032582 } \textit{ CC }
\item[] Patients with the CC genotype and cystic fibrosis may have increased clearance of dicloxacillin, when it is coadministered with cyclosporine, as compared to patients with the AA genotype. Other genetic and clinical factors may also influence clearance of dicloxacillin.
\end{rSubsection}

\end{rSection}\begin{rSection}{ cytarabine }
\item[]

\begin{rSubsection}{ ABCB1 }{ ATP-binding cassette, sub-family B (MDR/TAP), member 1 }{}{}
\item[]

\item[] ---------------------------------------------------- Clinical Annotations -----------------------------------------------------\newline
\item \textbf{\colorbox{teal} {Class 3}} \textbf{ rs1045642 } \textit{ GG }
\item[] Patients with the GG genotype may have 1) decreased exposure to doxorubicin metabolites and 2) decreased response to anthracycline regimens as compared to patients with the AA genotype, however the evidence is highly contradictory. Other genetic and clinical factors may also influence response to anthracycline regimens.\item \textbf{\colorbox{teal} {Class 3}} \textbf{ rs1128503 } \textit{ GG }
\item[] Patients with the GG genotype and acute myeloid leukemia may have a poorer response when treated with cytarabine, alone or in combination with daunorubicin, or dexrazoxane as compared to patients with the AA or AG genotype, however some evidence contradicts this. Other genetic and clinical factors may also influence response to cytarabine.
\end{rSubsection}

\end{rSection}\begin{rSection}{ daunorubicin }
\item[]

\begin{rSubsection}{ SLCO1B1 }{ solute carrier organic anion transporter family, member 1B1 }{}{}
\item[]

\item[] ---------------------------------------------------- Clinical Annotations -----------------------------------------------------\newline
\item \textbf{\colorbox{teal} {Class 3}} \textbf{ rs2291075 } \textit{ CT }
\item[] Patients with the CT genotype may have more favorable event-free and overall survival in children with de novo acute myeloid leukemia (AML) treated with cytarabine, daunorubicin, etoposide and mitoxantrone as compared to patients with genotype CC. Other genetic and clinical factors may also influence the treatment outcome in acute myeloid leukemia.
\end{rSubsection}

\end{rSection}\begin{rSection}{ deferasirox }
\item[]

\begin{rSubsection}{ UGT1A10 }{ UDP glucuronosyltransferase 1 family, polypeptide A10 }{}{}
\item[]

\item[] ---------------------------------------------------- Clinical Annotations -----------------------------------------------------\newline
\item \textbf{\colorbox{teal} {Class 3}} \textbf{ rs887829 } \textit{ CT }
\item[] Patients with the CT genotype and beta-thalassemia may have decreased concentrations of deferasirox as compared to patients with the TT genotype. Other genetic and clinical factors may also influence concentrations of deferasirox.\item \textbf{\colorbox{teal} {Class 3}} \textbf{ rs3806596 } \textit{ TC }
\item[] Patients with the CT genotype and beta-thalassemia may have worse response to, and decreased concentrations of deferasirox as compared to patients with the CC genotype. Other clinical and genetic factors may also influence concentrations of and response to deferasirox in patients with beta-thalassemia.\item \textbf{\colorbox{green} {Class 4}} \textbf{ rs4124874 } \textit{ TG }
\item[] Pediatric patients with major thalassemia and the GT genotype may have an increased risk of adverse reactions when administered deferasirox as compared to patients with the TT genotype. Please note, the evidence comes solely from a single case study report of a single individual, a 3 year old female patient with major thalassemia of genotype GT, therefore there is no information for patients with the GG or TT genotypes.Other clinical and genetic factors may also influence risk of adverse reactions in patients with major thalassemia who are administered deferasirox. \item \textbf{\colorbox{green} {Class 4}} \textbf{ rs10929302 } \textit{ GA }
\item[] Pediatric patients with major thalassemia and the AG genotype may have an increased risk of adverse reactions when administered deferasirox as compared to patients with the GG genotype. Please note, the evidence comes solely from a single case study report of a single individual, a 3 year old female patient with major thalassemia of genotype AG, therefore there is no information for patients with the AA or GG genotypes.Other clinical and genetic factors may also influence risk of adverse reactions in patients with major thalassemia who are administered deferasirox.\item \textbf{\colorbox{green} {Class 4}} \textbf{ rs4148323 } \textit{ GG }
\item[] Pediatric patients with major thalassemia and the GG genotype may have a decreased risk of adverse reactions when administered deferasirox as compared to patients with the AA or AG genotype. Please note, the evidence comes solely from a single case study report of a single individual, a 3 year old female patient with major thalassemia of genotype AG, therefore there is no information for patients with the GG or AA genotypes.Other clinical and genetic factors may also influence risk of adverse reactions in patients with major thalassemia who are administered deferasirox.
\end{rSubsection}

\end{rSection}\begin{rSection}{ diuretics }
\item[]

\begin{rSubsection}{ ADRB2 }{ adrenoceptor beta 2, surface }{}{}
\item[]

\item[] ---------------------------------------------------- Clinical Annotations -----------------------------------------------------\newline
\item \textbf{\colorbox{teal} {Class 3}} \textbf{ rs1042713 } \textit{ GG }
\item[] Patients with the GG genotype and heart failure may have increased emergency department visits and hospital utilization when treated with cardiovascular drugs as compared to patients with the AA or AG genotype. Other genetic and clinical factors may also influence efficacy of cardiovascular drugs.
\end{rSubsection}

\end{rSection}\begin{rSection}{ dobutamine }
\item[]

\begin{rSubsection}{ ADRB1 }{ adrenoceptor beta 1 }{}{}
\item[]

\item[] ---------------------------------------------------- Clinical Annotations -----------------------------------------------------\newline
\item \textbf{\colorbox{teal} {Class 3}} \textbf{ rs1801253 } \textit{ GG }
\item[] Healthy males with the GG genotype may have smaller increases in fractional shortening and systolic blood pressure when given dobutamine, as compared to healthy males with the CC genotype. No significant differences were seen for heart rate. Other genetic and clinical factors may also influence fractional shortening and systolic blood pressure.
\end{rSubsection}

\end{rSection}\begin{rSection}{ docetaxel }
\item[]

\begin{rSubsection}{ CYP3A4 }{ cytochrome P450, family 3, subfamily A, polypeptide 4 }{}{}
\item[]

\item[] ---------------------------------------------------- Clinical Annotations -----------------------------------------------------\newline
\item \textbf{\colorbox{teal} {Class 3}} \textbf{ rs2740574 } \textit{ TT }
\item[] Patients with the TT genotype may have decreased clearance of docetaxel and a decreased risk of an infusion-related reaction as compared to patients with the CC or CT genotype. These patients may experience a decreased risk of neurotoxicity with docetaxel treatment, though reports conflict. Other genetic and clinical factors may also influence clearance of and reactions to docetaxel.
\end{rSubsection}

\end{rSection}\begin{rSection}{ doxorubicin }
\item[]

\begin{rSubsection}{ ABCB1 }{ ATP-binding cassette, sub-family B (MDR/TAP), member 1 }{}{}
\item[]

\item[] ---------------------------------------------------- Clinical Annotations -----------------------------------------------------\newline
\item \textbf{\colorbox{teal} {Class 3}} \textbf{ rs1045642 } \textit{ GG }
\item[] Patients with the GG genotype may have 1) decreased exposure to doxorubicin metabolites and 2) decreased response to anthracycline regimens as compared to patients with the AA genotype, however the evidence is highly contradictory. Other genetic and clinical factors may also influence response to anthracycline regimens.\item \textbf{\colorbox{teal} {Class 3}} \textbf{ rs2032582 } \textit{ CC }
\item[] Patients with the  CC genotype may have increased metabolism of doxorubicin in people with Breast Neoplasms as compared to patients with genotype AA. Other genetic and clinical factors may also influence the metabolism of doxorubicin.
\end{rSubsection}

\end{rSection}\begin{rSection}{ efavirenz }
\item[]

\begin{rSubsection}{ ABCB1 }{ ATP-binding cassette, sub-family B (MDR/TAP), member 1 }{}{}
\item[]

\item[] ---------------------------------------------------- Clinical Annotations -----------------------------------------------------\newline
\item \textbf{\colorbox{teal} {Class 3}} \textbf{ rs1045642 } \textit{ GG }
\item[] Patients with the GG genotype and HIV infection who are treated with efavirenz may have reduced clearance of efavirenz as compared to patients with the AG genotype. Some studies have shown no association between this polymorphism and efavirenz clearance, plasma concentrations or exposure, or PBMC concentrations. Other genetic and clinical factors may also influence efavirenz pharmacokinetics. \item \textbf{\colorbox{teal} {Class 3}} \textbf{ rs2032582 } \textit{ CC }
\item[] Patients with the CC genotype may have increased likelihood of emerging viral drug resistance when exposed to efavirenz in people with HIV Infections as compared to patients with the AA genotype.This varaint is not associated with plasma exposure of efavirenz. Other genetic and clinical factors may also influence the response to efavirenz\item \textbf{\colorbox{green} {Class 4}} \textbf{ rs1128503 } \textit{ GG }
\item[] Patients with GG genotype and HIV may have increased concentrations of efavirenz in plasma compared to patients with AA genotype. However, this association was not significant and was not found in another study of plasma and PBMCs.  Other clinical and genetic factors may affect efavirenz concentrations.
\end{rSubsection}

\end{rSection}\begin{rSection}{ enalapril }
\item[]

\begin{rSubsection}{ ADRB2 }{ adrenoceptor beta 2, surface }{}{}
\item[]

\item[] ---------------------------------------------------- Clinical Annotations -----------------------------------------------------\newline
\item \textbf{\colorbox{teal} {Class 3}} \textbf{ rs1042714 } \textit{ GG }
\item[] Patients with the GG genotype and left ventricular hypertrophy may have a greater percent reduction in left ventricular mass index when treated with enalapril as compared to patients with the CC genotype. Other genetic and clinical factors may also influence reduction in left ventricular mass index. 
\end{rSubsection}

\end{rSection}\begin{rSection}{ erlotinib }
\item[]

\begin{rSubsection}{ CYP1A2 }{ cytochrome P450, family 1, subfamily A, polypeptide 2 }{}{}
\item[]

\item[] ---------------------------------------------------- Clinical Annotations -----------------------------------------------------\newline
\item \textbf{\colorbox{teal} {Class 3}} \textbf{ rs2472304 } \textit{ AA }
\item[] Patients with the AA genotype may have increased concentrations of erlotinib as compared to patients with the GG genotype. Other genetic and clinical factors may also influence concentrations of erlotinib.
\end{rSubsection}

\end{rSection}\begin{rSection}{ escitalopram }
\item[]

\begin{rSubsection}{ CYP2D6 }{ cytochrome P450, family 2, subfamily D, polypeptide 6 }{}{}
\item[]

\item[] ---------------------------------------------------- Clinical Annotations -----------------------------------------------------\newline
\item \textbf{\colorbox{teal} {Class 3}} \textbf{ rs1065852 } \textit{ GG }
\item[] Patients with the GG genotype and depression may have a increased response and remission rate when treated with escitalopram as compared to patients with the AA genotype. Other genetic and clinical factors may also effect patients response.
\end{rSubsection}

\end{rSection}\begin{rSection}{ ethambutol }
\item[]

\begin{rSubsection}{ NAT2 }{ N-acetyltransferase 2 (arylamine N-acetyltransferase) }{}{}
\item[]

\item[] ---------------------------------------------------- Clinical Annotations -----------------------------------------------------\newline
\item \textbf{\colorbox{cyan} {Class 2A}} \textbf{ rs1041983 } \textit{ TT }
\item[] Patients with the TT genotype and tuberculosis (TB) may have an increased risk for hepatotoxicity when treated with anti-TB drugs as compared to patients with the CC genotype. Other genetic and clinical factors may also influence risk for hepatotoxicity.\item \textbf{\colorbox{cyan} {Class 2A}} \textbf{ rs1799930 } \textit{ AA }
\item[] Patients with the AA genotype and tuberculosis (TB) may have an increased risk of hepatotoxicity when treated with anti-TB drugs as compared to patients with the GG genotype. They also may have decreased clearance of isoniazid as compared to those with the AG or GG genotype. Other genetic and clinical factors may also influence risk for hepatotoxicity and clearance of isoniazid.\item \textbf{\colorbox{teal} {Class 3}} \textbf{ rs1799931 } \textit{ GG }
\item[] Patients with the GG genotype and tuberculosis (TB) may have a decreased risk of hepatotoxicity when treated with anti-TB drugs as compared to patients with the AA or AG genotype. However, some studies find no association with hepatotoxicity. Other genetic and clinical factors may also influence risk of hepatotoxicity.
\end{rSubsection}

\end{rSection}\begin{rSection}{ ethanol }
\item[]

\begin{rSubsection}{ CHRNA3 }{ cholinergic receptor, nicotinic, alpha 3 (neuronal) }{}{}
\item[]

\item[] ---------------------------------------------------- Clinical Annotations -----------------------------------------------------\newline
\item \textbf{\colorbox{teal} {Class 3}} \textbf{ rs16969968 } \textit{ GG }
\item[] Patients with the GG genotype may have an increased risk for alcoholism as compared to patients with the AA genotype. Other genetic and clinical factors may also influence risk of alcoholism.
\end{rSubsection}

\end{rSection}\begin{rSection}{ etoposide }
\item[]

\begin{rSubsection}{ SLCO1B1 }{ solute carrier organic anion transporter family, member 1B1 }{}{}
\item[]

\item[] ---------------------------------------------------- Clinical Annotations -----------------------------------------------------\newline
\item \textbf{\colorbox{teal} {Class 3}} \textbf{ rs2291075 } \textit{ CT }
\item[] Patients with the CT genotype may have more favorable event-free and overall survival in children with de novo acute myeloid leukemia (AML) treated with cytarabine, daunorubicin, etoposide and mitoxantrone as compared to patients with genotype CC. Other genetic and clinical factors may also influence the treatment outcome in acute myeloid leukemia.
\end{rSubsection}

\end{rSection}\begin{rSection}{ fexofenadine }
\item[]

\begin{rSubsection}{ ABCB1 }{ ATP-binding cassette, sub-family B (MDR/TAP), member 1 }{}{}
\item[]

\item[] ---------------------------------------------------- Clinical Annotations -----------------------------------------------------\newline
\item \textbf{\colorbox{teal} {Class 3}} \textbf{ rs1045642 } \textit{ GG }
\item[] Healthy individuals with the GG genotype who are treated with fexofenadine may have higher plasma drug levels as compared with healthy individuals with the AA genotype. Another study found no association with fexofenadine plasma concentrations. Other genetic and clinical factors may also influence plasma concentrations of fexofenadine and dose requirements.
\end{rSubsection}

\end{rSection}\begin{rSection}{ fluorouracil }
\item[]

\begin{rSubsection}{ ABCB1 }{ ATP-binding cassette, sub-family B (MDR/TAP), member 1 }{}{}
\item[]

\item[] ---------------------------------------------------- Clinical Annotations -----------------------------------------------------\newline
\item \textbf{\colorbox{teal} {Class 3}} \textbf{ rs1045642 } \textit{ GG }
\item[] Patients with GG genotype may have decreased risk of diarrhea when treated with fluorouracil in people with Colorectal Neoplasms as compared to patients with genotype AA. Other genetic and clinical factors may also impact a patients response to fluorouracil.
\end{rSubsection}

\end{rSection}\begin{rSection}{ fluvastatin }
\item[]

\begin{rSubsection}{ SLCO1B1 }{ solute carrier organic anion transporter family, member 1B1 }{}{}
\item[]

\item[] ---------------------------------------------------- Clinical Annotations -----------------------------------------------------\newline
\item \textbf{\colorbox{teal} {Class 3}} \textbf{ rs11045819 } \textit{ CC }
\item[] Patients with the CC genotype who are treated with fluvastatin may have a lesser reduction in LDL-C as compared to patients with the AC and AA genotype.
\end{rSubsection}

\end{rSection}\begin{rSection}{ gefitinib }
\item[]

\begin{rSubsection}{ ABCB1 }{ ATP-binding cassette, sub-family B (MDR/TAP), member 1 }{}{}
\item[]

\item[] ---------------------------------------------------- Clinical Annotations -----------------------------------------------------\newline
\item \textbf{\colorbox{teal} {Class 3}} \textbf{ rs1128503 } \textit{ GG }
\item[] Patients with the GG genotype and non-small cell lung cancer may have a decreased risk for diarrhea and skin rash when treated with gefitinib as compared to patients with the AA genotype. Other genetic and clinical factors may also influence drug toxicity risk in patients receiving gefitinib.
\end{rSubsection}

\end{rSection}\begin{rSection}{ gemcitabine }
\item[]

\begin{rSubsection}{ CDA }{ cytidine deaminase }{}{}
\item[]

\item[] ---------------------------------------------------- Clinical Annotations -----------------------------------------------------\newline
\item \textbf{\colorbox{teal} {Class 3}} \textbf{ rs1048977 } \textit{ CT }
\item[] Patients with cancer and the CT genotype may have increased metabolism of gemcitabine as compared to patients with the TT genotype. However, this has been contradicted by some studies. Other genetic and clinical factors may also influence metabolism of gemcitabine. 
\end{rSubsection}

\end{rSection}\begin{rSection}{ haloperidol }
\item[]

\begin{rSubsection}{ COMT }{ catechol-O-methyltransferase }{}{}
\item[]

\item[] ---------------------------------------------------- Clinical Annotations -----------------------------------------------------\newline
\item \textbf{\colorbox{teal} {Class 3}} \textbf{ rs4680 } \textit{ GA }
\item[] Patients with the AG genotype and schizophrenia may have an increased risk for developing extrapyramidal symptoms when treated with haloperidol as compared to patients with the AA or GG genotype. Other genetic and clinical factors may also influence risk for extrapyramidal symptoms when taking haloperidol.
\end{rSubsection}

\end{rSection}\begin{rSection}{ hmg coa reductase inhibitors }
\item[]

\begin{rSubsection}{ ABCB1 }{ ATP-binding cassette, sub-family B (MDR/TAP), member 1 }{}{}
\item[]

\item[] ---------------------------------------------------- Clinical Annotations -----------------------------------------------------\newline
\item \textbf{\colorbox{teal} {Class 3}} \textbf{ rs1045642 } \textit{ GG }
\item[] Patients with the GG genotype may have decreased serum creatine kinase levels when treated with hmg CoA reductase inhibitors as compared to patients with the AA genotype. Other genetic and clinical factors may also influence serum creatine kinase levels. \item \textbf{\colorbox{teal} {Class 3}} \textbf{ rs1128503 } \textit{ GG }
\item[] Patients with the GG genotype may have decreased serum creatine kinase levels when treated with hmg CoA reductase inhibitors as compared to patients with the AA genotypes. Other genetic and clinical factors may also influence serum creatine kinase levels. 
\end{rSubsection}

\end{rSection}\begin{rSection}{ iloperidone }
\item[]

\begin{rSubsection}{ CYP2D6 }{ cytochrome P450, family 2, subfamily D, polypeptide 6 }{}{}
\item[]

\item[] ---------------------------------------------------- Clinical Annotations -----------------------------------------------------\newline
\item \textbf{\colorbox{teal} {Class 3}} \textbf{ rs1065852 } \textit{ GG }
\item[] Patients with the GG genotype and schizophrenia may have an increased QTc interval when treated with iloperidone as compared to patients with the AA or AG genotype. Other genetic and clinical factors may also influence QTc interval.
\end{rSubsection}

\end{rSection}\begin{rSection}{ imatinib }
\item[]

\begin{rSubsection}{ ABCB1 }{ ATP-binding cassette, sub-family B (MDR/TAP), member 1 }{}{}
\item[]

\item[] ---------------------------------------------------- Clinical Annotations -----------------------------------------------------\newline
\item \textbf{\colorbox{teal} {Class 3}} \textbf{ rs1045642 } \textit{ GG }
\item[] Patients with the GG genotype and chronic myeloid leukemia may have an increased likelihood of achieving complete molecular response when treated with imatinib, as compared to patients with the AA or AG genotype. However, this was only significant in an exclusively Caucasian population. Additionally, no significant results were seen when considering major molecular response. Other genetic and clinical factors may also influence likelihood of achieving complete molecular response.\item \textbf{\colorbox{teal} {Class 3}} \textbf{ rs1128503 } \textit{ GG }
\item[] Patients with the GG genotype and chronic myeloid leukemia may have a better response to imatinib treatment as compared to patients with the AA or AG genotype. Other genetic and clinical factors may also influence response to imatinib. 
\end{rSubsection}

\end{rSection}\begin{rSection}{ irbesartan }
\item[]

\begin{rSubsection}{ CYP2C9 }{ cytochrome P450, family 2, subfamily C, polypeptide 9 }{}{}
\item[]

\item[] ---------------------------------------------------- Clinical Annotations -----------------------------------------------------\newline
\item \textbf{\colorbox{teal} {Class 3}} \textbf{ rs1057910 } \textit{ AC }
\item[] Patients with the AC genotype and essential hypertension may have decreased metabolism or clearance of irbesartan as compared to patients with the AA genotype, but may have no difference in response. Other clinical or genetic factors may also influence concentrations of irbesartan in patients with essential hypertension. \item \textbf{\colorbox{teal} {Class 3}} \textbf{ rs72558187 } \textit{ TT }
\item[] Individuals with the TT genotype may have increased metabolism and clearance of irbesartan which may result may in decreased exposure of irbesartan as compared to patients with the CT genotype. Other clinical and genetic factors may also influence metabolism of irbesartan.
\end{rSubsection}

\end{rSection}\begin{rSection}{ irinotecan }
\item[]

\begin{rSubsection}{ SLCO1B1 }{ solute carrier organic anion transporter family, member 1B1 }{}{}
\item[]

\item[] ---------------------------------------------------- Clinical Annotations -----------------------------------------------------\newline
\item \textbf{\colorbox{teal} {Class 3}} \textbf{ rs4149056 } \textit{ TT }
\item[] Patients with the TT genotype and cancer may have a decreased risk of neutropenia when treated with irinotecan or irinotecan-based regimens, as compared to patients with the CC or CT genotype. However, a different study of similar size found no association between the TT genotype and neutropenia. No significant results have been seen for diarrhea. Other genetic and clinical factors may also influence risk of neutropenia or diarrhea.\item \textbf{\colorbox{teal} {Class 3}} \textbf{ rs4149015 } \textit{ GG }
\item[] Patients with the GG genotype and non-small cell lung cancer may have a decreased risk of neutropenia when treated with irinotecan as compared to patients with the AG or GG genotype. No association has been seen for diarrhea. Other genetic and clinical factors may also influence risk of neutropenia.\item \textbf{\colorbox{teal} {Class 3}} \textbf{ rs2306283 } \textit{ AG }
\item[] Patients with the AG genotype and solid tumors may experience increased risk of neutropenia compared to patients with the AA genotype. However, studies conflict as to this association. Other clinical and genetic factors may affect risk of neutropenia with irinotecan therapy.
\end{rSubsection}

\end{rSection}\begin{rSection}{ isoniazid }
\item[]

\begin{rSubsection}{ NAT2 }{ N-acetyltransferase 2 (arylamine N-acetyltransferase) }{}{}
\item[]

\item[] ---------------------------------------------------- Clinical Annotations -----------------------------------------------------\newline
\item \textbf{\colorbox{cyan} {Class 2A}} \textbf{ rs1041983 } \textit{ TT }
\item[] Patients with the TT genotype and tuberculosis (TB) may have an increased risk for hepatotoxicity when treated with anti-TB drugs as compared to patients with the CC genotype. Other genetic and clinical factors may also influence risk for hepatotoxicity.\item \textbf{\colorbox{cyan} {Class 2A}} \textbf{ rs1799930 } \textit{ AA }
\item[] Patients with the AA genotype and tuberculosis (TB) may have an increased risk of hepatotoxicity when treated with anti-TB drugs as compared to patients with the GG genotype. They also may have decreased clearance of isoniazid as compared to those with the AG or GG genotype. Other genetic and clinical factors may also influence risk for hepatotoxicity and clearance of isoniazid.\item \textbf{\colorbox{teal} {Class 3}} \textbf{ rs1799931 } \textit{ GG }
\item[] Patients with the GG genotype and tuberculosis (TB) may have a decreased risk of hepatotoxicity when treated with anti-TB drugs as compared to patients with the AA or AG genotype. However, some studies find no association with hepatotoxicity. Other genetic and clinical factors may also influence risk of hepatotoxicity.
\end{rSubsection}

\end{rSection}\begin{rSection}{ ivacaftor }
\item[]

\begin{rSubsection}{ CFTR }{ cystic fibrosis transmembrane conductance regulator (ATP-binding cassette sub-family C, member 7) }{}{}
\item[]

\item[] ---------------------------------------------------- Clinical Annotations -----------------------------------------------------\newline
\item \textbf{\colorbox{red} {Class 1A}} \textbf{ rs78655421 } \textit{ GG }
\item[] Patients with the GG genotype and cystic fibrosis may not respond when treated with ivacaftor as compared to patients with the AA and AG genotypes. Other genetic and clinical factors may also influence the efficacy of ivacaftor.
\end{rSubsection}

\end{rSection}\begin{rSection}{ lamivudine }
\item[]

\begin{rSubsection}{ ABCB1 }{ ATP-binding cassette, sub-family B (MDR/TAP), member 1 }{}{}
\item[]

\item[] ---------------------------------------------------- Clinical Annotations -----------------------------------------------------\newline
\item \textbf{\colorbox{teal} {Class 3}} \textbf{ rs1045642 } \textit{ GG }
\item[] Patients with the GG genotype and HIV may have an increased risk of virological failure when receiving highly active antiretroviral therapy (HAART), as compared to patients with the AA genotype. Other genetic and clinical factors may also influence risk of virological failure on HAART.
\end{rSubsection}

\end{rSection}\begin{rSection}{ lamotrigine }
\item[]

\begin{rSubsection}{ UGT1A10 }{ UDP glucuronosyltransferase 1 family, polypeptide A10 }{}{}
\item[]

\item[] ---------------------------------------------------- Clinical Annotations -----------------------------------------------------\newline
\item \textbf{\colorbox{blue} {Class 2B}} \textbf{ rs2011425 } \textit{ TT }
\item[] Patients with the TT genotype and epilepsy who are administered lamotrigine may have increased serum concentrations of lamotrigine, as well as improved response to lamotrigine, and may need a higher dose as compared to patients with the GG genotype. Other clinical and genetic factors may also influence metabolism, response, and dose of lamotrigine.  
\end{rSubsection}

\end{rSection}\begin{rSection}{ lorazepam }
\item[]

\begin{rSubsection}{ UGT2B15 }{ UDP glucuronosyltransferase 2 family, polypeptide B15 }{}{}
\item[]

\item[] ---------------------------------------------------- Clinical Annotations -----------------------------------------------------\newline
\item \textbf{\colorbox{blue} {Class 2B}} \textbf{ rs1902023 } \textit{ AC }
\item[] Subjects with the AC genotype may have decreased clearance of oxazepam or lorazepam as compared to subjects with the CC genotype, or increased clearance as compared to subjects with the AA genotype. Other genetic and clinical factors may also influence the oral clearance of oxazepam or lorazepam.
\end{rSubsection}

\end{rSection}\begin{rSection}{ losartan }
\item[]

\begin{rSubsection}{ CYP2C9 }{ cytochrome P450, family 2, subfamily C, polypeptide 9 }{}{}
\item[]

\item[] ---------------------------------------------------- Clinical Annotations -----------------------------------------------------\newline
\item \textbf{\colorbox{teal} {Class 3}} \textbf{ rs1057910 } \textit{ AC }
\item[] Subjects with the AC genotype who are treated with losartan may have decreased metabolism of losartan as compared to subjects with the AA genotype. Other genetic and clinical factors may also influence metabolism of losartan.
\end{rSubsection}

\end{rSection}\begin{rSection}{ mercaptopurine }
\item[]

\begin{rSubsection}{ TPMT }{ thiopurine S-methyltransferase }{}{}
\item[]
\item[] ------------------------------------------------------ Dosing Guideline --------------------------------------------------------\newline
\item[]
\item[] \textbf{ TPMT:*1/*1 } Strong

\item[] Start with normal starting dose (e.g., 75 mg/m2/d or 1.5 mg/kg/d) and adjust doses of mercaptopurine (and of any other myelosuppressive therapy) without any special emphasis on mercaptopurine compared to other agents. Allow 2 weeks to reach steady state after each dose adjustment.
\item[] ---------------------------------------------------- Clinical Annotations -----------------------------------------------------\newline
\item \textbf{\colorbox{teal} {Class 3}} \textbf{ rs1142345 } \textit{ TT }
\item[] Pediatric patients with the TT genotype and Precursor Cell Lymphoblastic Leukemia-Lymphoma may experience decreased GI toxicity when treated with mercaptopurine and may require an increased dose as compared to patients with the CT or CC genotypes. Other genetic and clinical factors may also influence the likelihood of GI toxicity and dose of mercaptopurine in pediatric patients with Precursor Cell Lymphoblastic Leukemia-Lymphoma.\item \textbf{\colorbox{green} {Class 4}} \textbf{ rs3931660 } \textit{ AA }
\item[] Patients with the AA genotype may have increased TPMT activity toward mercaptopurine as compared to patients with the AT genotype. Other genetic and clinical factors may also influence TPMT activity. 
\end{rSubsection}

\end{rSection}\begin{rSection}{ methadone }
\item[]

\begin{rSubsection}{ CYP2B6 }{ cytochrome P450, family 2, subfamily B, polypeptide 6 }{}{}
\item[]

\item[] ---------------------------------------------------- Clinical Annotations -----------------------------------------------------\newline
\item \textbf{\colorbox{cyan} {Class 2A}} \textbf{ rs3745274 } \textit{ GG }
\item[] Patients with the GG genotype who are being treated with methadone for heroin addiction may require an increased dose of the drug as compared to patients with the TT genotype. Other genetic and clinical factors may also influence dose of methadone.\item \textbf{\colorbox{teal} {Class 3}} \textbf{ rs2279343 } \textit{ AA }
\item[] Patients with the AA genotype who are being treated with methadone for heroin addiction may require an increased dose of the drug as compared to patients with the GG genotype. However, one study found no association between this variant and methadone dose. Other genetic and clinical factors may also influence dose of methadone.
\end{rSubsection}

\end{rSection}\begin{rSection}{ methotrexate }
\item[]

\begin{rSubsection}{ SLCO1B1 }{ solute carrier organic anion transporter family, member 1B1 }{}{}
\item[]

\item[] ---------------------------------------------------- Clinical Annotations -----------------------------------------------------\newline
\item \textbf{\colorbox{teal} {Class 3}} \textbf{ rs4149056 } \textit{ TT }
\item[] Pediatric patients with the TT genotype and acute lymphoblastic leukemia may have increased clearance of methotrexate as compared to patients with the CC or CT genotype. Other genetic and clinical factors may also influence clearance of methotrexate.\item \textbf{\colorbox{teal} {Class 3}} \textbf{ rs2306283 } \textit{ AG }
\item[] Pediatric patients with the AG genotype and acute lymphoblastic leukemia may have increased clearance of methotrexate as compared to patients with the GG genotype. Other genetic and clinical factors may also influence clearance of methotrexate. 
\end{rSubsection}

\end{rSection}\begin{rSection}{ metoprolol }
\item[]

\begin{rSubsection}{ ADRB2 }{ adrenoceptor beta 2, surface }{}{}
\item[]

\item[] ---------------------------------------------------- Clinical Annotations -----------------------------------------------------\newline
\item \textbf{\colorbox{teal} {Class 3}} \textbf{ rs1042714 } \textit{ GG }
\item[] Patients with the GG genotype and hypertension may have an increased risk of developing hypertriglyceridemia when treated with atenolol or metoprolol as compared to patients with the CC or CG genotype. Other genetic and clinical factors may also influence risk of hypertriglyceridemia. 
\end{rSubsection}

\end{rSection}\begin{rSection}{ midazolam }
\item[]

\begin{rSubsection}{ CYP3A4 }{ cytochrome P450, family 3, subfamily A, polypeptide 4 }{}{}
\item[]

\item[] ---------------------------------------------------- Clinical Annotations -----------------------------------------------------\newline
\item \textbf{\colorbox{teal} {Class 3}} \textbf{ rs35599367 } \textit{ GG }
\item[] Patients with the GG genotype and tumors may have increased metabolism of midazolam as compared to patients with the AG genotype. Other genetic and clinical factors may also influence metabolism of midazolam. \item \textbf{\colorbox{green} {Class 4}} \textbf{ rs12721627 } \textit{ GG }
\item[] The expression of a construct caring the G variant is not associated with decreased clearance of midazolam in transfected cells.
\end{rSubsection}

\end{rSection}\begin{rSection}{ morphine }
\item[]

\begin{rSubsection}{ ABCB1 }{ ATP-binding cassette, sub-family B (MDR/TAP), member 1 }{}{}
\item[]

\item[] ---------------------------------------------------- Clinical Annotations -----------------------------------------------------\newline
\item \textbf{\colorbox{teal} {Class 3}} \textbf{ rs1045642 } \textit{ GG }
\item[] Patients with the GG genotype may have decreased pain reduction when treated with morphine in cancer patients as compared to patients with genotype AA. Other genetic and clinical factors may also influence response to morphine.
\end{rSubsection}

\end{rSection}\begin{rSection}{ nevirapine }
\item[]

\begin{rSubsection}{ CYP2B6 }{ cytochrome P450, family 2, subfamily B, polypeptide 6 }{}{}
\item[]

\item[] ---------------------------------------------------- Clinical Annotations -----------------------------------------------------\newline
\item \textbf{\colorbox{cyan} {Class 2A}} \textbf{ rs3745274 } \textit{ GG }
\item[] Patients with the GG genotype and HIV infection may have increased clearance of and decreased exposure to nevirapine as compared to patients with the TT or GT genotype. Other genetic and clinical factors may also influence clearance of nevirapine and exposure to drug.\item \textbf{\colorbox{teal} {Class 3}} \textbf{ rs28399499 } \textit{ TT }
\item[] Patients with the TT genotype and HIV may have a decreased risk for Stevens-Johnson Syndrome/toxic epidermal necrolysis (SJS/TEN) when treated with nevirapine as compared to patients with the CC or CT genotype. Other genetic and clinical factors may also influence risk for developing SJS/TEN when receiving nevirapine.
\end{rSubsection}

\end{rSection}\begin{rSection}{ nicotine }
\item[]

\begin{rSubsection}{ CYP2A6 }{ cytochrome P450, family 2, subfamily A, polypeptide 6 }{}{}
\item[]

\item[] ---------------------------------------------------- Clinical Annotations -----------------------------------------------------\newline
\item \textbf{\colorbox{green} {Class 4}} \textbf{ rs5031016 } \textit{ AA }
\item[] Patients with the AA genotype may have increased metabolism of nicotine as compared to patients with the GG or AG genotype. Other variants within the CYP2A6 gene should be considered - allele G of this SNP is part of the *7, *10, *19, *36, *37 CYP2A6 alleles. Other genetic and clinical factors may also influence metabolism of nicotine.
\end{rSubsection}

\end{rSection}\begin{rSection}{ nifedipine }
\item[]

\begin{rSubsection}{ CYP3A4 }{ cytochrome P450, family 3, subfamily A, polypeptide 4 }{}{}
\item[]

\item[] ---------------------------------------------------- Clinical Annotations -----------------------------------------------------\newline
\item \textbf{\colorbox{green} {Class 4}} \textbf{ rs4987161 } \textit{ AA }
\item[] In vitro, the construct expressing the wild type allelic protein has average nifedipine metabolism.
\end{rSubsection}

\end{rSection}\begin{rSection}{ ondansetron }
\item[]

\begin{rSubsection}{ CYP3A5 }{ cytochrome P450, family 3, subfamily A, polypeptide 5 }{}{}
\item[]

\item[] ---------------------------------------------------- Clinical Annotations -----------------------------------------------------\newline
\item \textbf{\colorbox{teal} {Class 3}} \textbf{ rs776746 } \textit{ CC }
\item[] Patients with the CC genotype may have increased metabolism of ondansetron as compared to patients with the TT genotype. Other genetic and clinical factors may also influence metabolism of ondansetron.
\end{rSubsection}

\end{rSection}\begin{rSection}{ opioids }
\item[]

\begin{rSubsection}{ ABCB1 }{ ATP-binding cassette, sub-family B (MDR/TAP), member 1 }{}{}
\item[]

\item[] ---------------------------------------------------- Clinical Annotations -----------------------------------------------------\newline
\item \textbf{\colorbox{teal} {Class 3}} \textbf{ rs1045642 } \textit{ GG }
\item[] Patients with the GG genotype may have a decreased risk of opioid dependence when exposed to opioids as compared to patients with the AG genotype. Other clinical and genetic factors may also influence risk of opioid dependence upon exposure to opioids. 
\end{rSubsection}

\end{rSection}\begin{rSection}{ paclitaxel }
\item[]

\begin{rSubsection}{ CYP2C8 }{ cytochrome P450, family 2, subfamily C, polypeptide 8 }{}{}
\item[]

\item[] ---------------------------------------------------- Clinical Annotations -----------------------------------------------------\newline
\item \textbf{\colorbox{green} {Class 4}} \textbf{ rs11572103 } \textit{ TT }
\item[] Patients with the TT genotype may have increased clearance of paclitaxel as compared to patients with the AA or AT genotypes, however this has not been shown in vivo. Other genetic and clinical factors may also influence clearance of paclitaxel.\item \textbf{\colorbox{green} {Class 4}} \textbf{ rs11572080 } \textit{ CC }
\item[] Patients with the CC genotype may have increased clearance of paclitaxel as compared to patients with the CT or TT genotypes, however this has not been shown in vivo. Other genetic and clinical factors may also influence clearance of paclitaxel.\item \textbf{\colorbox{green} {Class 4}} \textbf{ rs10509681 } \textit{ TT }
\item[] Patients with the TT genotype may have increased metabolism of paclitaxel as compared to patients with the CT or CC genotypes, however this has not been shown in vivo. Other genetic and clinical factors may also influence clearance of paclitaxel.
\end{rSubsection}

\end{rSection}\begin{rSection}{ peginterferon alfa-2a }
\item[]

\begin{rSubsection}{ IFNL3 }{ interferon, lambda 3 }{}{}
\item[]
\item[] ------------------------------------------------------ Dosing Guideline --------------------------------------------------------\newline
\item[]
\item[] \textbf{ IFNL3:rs12979860C/rs12979860C } Strong
\item[] Phenotype (Genotype)\newline
\item[] 
\item[] ---------------------------------------------------- Clinical Annotations -----------------------------------------------------\newline
\item \textbf{\colorbox{teal} {Class 3}} \textbf{ rs12979860 } \textit{ CT }
\item[] Patients with genotype CT may have decreased response to daclatasvir, peginterferon alfa-2a, peginterferon alfa-2b and ribavirin in people with Hepatitis C, Chronic as compared to genotypes CC. SVR24 rates are higher in patients treated with the combination of daclatasvir and pegIFN-alfa/RBV than those receiving pegIFN-alfa/RBV alone across all IL28B genotypes (CC, CT, or TT) regardless of viral subtypes. Other genetic and clinical factors may also influence the response to daclatasvir therapy.
\end{rSubsection}

\end{rSection}\begin{rSection}{ peginterferon alfa-2b }
\item[]

\begin{rSubsection}{ IFNL3 }{ interferon, lambda 3 }{}{}
\item[]
\item[] ------------------------------------------------------ Dosing Guideline --------------------------------------------------------\newline
\item[]
\item[] \textbf{ IFNL3:rs12979860C/rs12979860C } Strong
\item[] Phenotype (Genotype)\newline
\item[] 
\item[] ---------------------------------------------------- Clinical Annotations -----------------------------------------------------\newline
\item \textbf{\colorbox{teal} {Class 3}} \textbf{ rs12979860 } \textit{ CT }
\item[] Patients with genotype CT may have decreased response to daclatasvir, peginterferon alfa-2a, peginterferon alfa-2b and ribavirin in people with Hepatitis C, Chronic as compared to genotypes CC. SVR24 rates are higher in patients treated with the combination of daclatasvir and pegIFN-alfa/RBV than those receiving pegIFN-alfa/RBV alone across all IL28B genotypes (CC, CT, or TT) regardless of viral subtypes. Other genetic and clinical factors may also influence the response to daclatasvir therapy.
\end{rSubsection}

\end{rSection}\begin{rSection}{ phenprocoumon }
\item[]

\begin{rSubsection}{ VKORC1 }{ vitamin K epoxide reductase complex, subunit 1 }{}{}
\item[]

\item[] ---------------------------------------------------- Clinical Annotations -----------------------------------------------------\newline
\item \textbf{\colorbox{cyan} {Class 2A}} \textbf{ rs9934438 } \textit{ GA }
\item[] Patients with the AG genotype may have decreased dose of acenocoumarol or phenprocoumon as compared to patients with genotype GG. Other genetic and clinical factors may also influence the dose of acenocoumarol or phenprocoumon.
\end{rSubsection}

\end{rSection}\begin{rSection}{ phenytoin }
\item[]

\begin{rSubsection}{ SCN1A }{ sodium channel, voltage-gated, type I, alpha subunit }{}{}
\item[]

\item[] ---------------------------------------------------- Clinical Annotations -----------------------------------------------------\newline
\item \textbf{\colorbox{blue} {Class 2B}} \textbf{ rs3812718 } \textit{ CT }
\item[] Patients with the CT genotype who are treated with phenytoin may require a higher dose as compared to patients with the CC genotype. Other genetic and clinical factors may also influence dose of phenytoin.
\end{rSubsection}

\end{rSection}\begin{rSection}{ pravastatin }
\item[]

\begin{rSubsection}{ LDLR }{ low density lipoprotein receptor }{}{}
\item[]

\item[] ---------------------------------------------------- Clinical Annotations -----------------------------------------------------\newline
\item \textbf{\colorbox{teal} {Class 3}} \textbf{ rs1433099 } \textit{ TC }
\item[] Patients with the CT genotype and vascular diseases may have a poorer response to pravastatin treatment as compared to patients with the TT genotype, or a better response as compared to patients with the CC genotype. Other genetic and clinical factors may also influence pravastatin response. 
\end{rSubsection}

\end{rSection}\begin{rSection}{ propofol }
\item[]

\begin{rSubsection}{ UGT1A10 }{ UDP glucuronosyltransferase 1 family, polypeptide A10 }{}{}
\item[]

\item[] ---------------------------------------------------- Clinical Annotations -----------------------------------------------------\newline
\item \textbf{\colorbox{green} {Class 4}} \textbf{ rs58597806 } \textit{ GG }
\item[] Patients with the GG genotype may have decreased but not non-existent risk of adverse effects when treated with propofol as compared to patients with the AA or AG genotype. Other genetic and clinical factors may also influence response to propofol.
\end{rSubsection}

\end{rSection}\begin{rSection}{ quetiapine }
\item[]

\begin{rSubsection}{ COMT }{ catechol-O-methyltransferase }{}{}
\item[]

\item[] ---------------------------------------------------- Clinical Annotations -----------------------------------------------------\newline
\item \textbf{\colorbox{teal} {Class 3}} \textbf{ rs6269 } \textit{ AG }
\item[] Patients with the AG genotype and schizophrenia may have a poorer response to treatment with quetiapine as compared to patients with the GG genotype, or a better response as compared to patients with the AA genotype. Other genetic and clinical factors may also influence quetiapine response. \item \textbf{\colorbox{teal} {Class 3}} \textbf{ rs4818 } \textit{ CG }
\item[] Patients with the CG genotype and schizophrenia may have a poorer response to treatment with quetiapine as compared to patients with the GG genotype, or a better response as compared to patients with the CC genotype. Other genetic and clinical factors may also influence quetiapine response. 
\end{rSubsection}

\end{rSection}\begin{rSection}{ raloxifene }
\item[]

\begin{rSubsection}{ UGT1A8 }{ UDP glucuronosyltransferase 1 family, polypeptide A8 }{}{}
\item[]

\item[] ---------------------------------------------------- Clinical Annotations -----------------------------------------------------\newline
\item \textbf{\colorbox{teal} {Class 3}} \textbf{ rs1042597 } \textit{ CC }
\item[] Post menopausal women with the CC genotype and schizophrenia may have increased response to raloxifene compared to patients with the CG genotype. Other genetic and clinical factors may affect response to raloxifene. 
\end{rSubsection}

\end{rSection}\begin{rSection}{ ranibizumab }
\item[]

\begin{rSubsection}{ VEGFA }{ vascular endothelial growth factor A }{}{}
\item[]

\item[] ---------------------------------------------------- Clinical Annotations -----------------------------------------------------\newline
\item \textbf{\colorbox{teal} {Class 3}} \textbf{ rs2010963 } \textit{ CG }
\item[] Patients with the CG genotype and choroidal neovascularization may have a better response to anti-VEGF treatment, as compared to patients with the CC genotype. Other genetic and clinical factors may also influence response to anti-VEGF treatment. 
\end{rSubsection}

\end{rSection}\begin{rSection}{ repaglinide }
\item[]

\begin{rSubsection}{ SLCO1B1 }{ solute carrier organic anion transporter family, member 1B1 }{}{}
\item[]

\item[] ---------------------------------------------------- Clinical Annotations -----------------------------------------------------\newline
\item \textbf{\colorbox{teal} {Class 3}} \textbf{ rs2306283 } \textit{ AG }
\item[] While the GG genotype is associated with reduced plasma concentrations of repaglinide, no results are shown for the GA genotype.
\end{rSubsection}

\end{rSection}\begin{rSection}{ rifampin }
\item[]

\begin{rSubsection}{ NAT2 }{ N-acetyltransferase 2 (arylamine N-acetyltransferase) }{}{}
\item[]

\item[] ---------------------------------------------------- Clinical Annotations -----------------------------------------------------\newline
\item \textbf{\colorbox{cyan} {Class 2A}} \textbf{ rs1041983 } \textit{ TT }
\item[] Patients with the TT genotype and tuberculosis (TB) may have an increased risk for hepatotoxicity when treated with anti-TB drugs as compared to patients with the CC genotype. Other genetic and clinical factors may also influence risk for hepatotoxicity.\item \textbf{\colorbox{cyan} {Class 2A}} \textbf{ rs1799930 } \textit{ AA }
\item[] Patients with the AA genotype and tuberculosis (TB) may have an increased risk of hepatotoxicity when treated with anti-TB drugs as compared to patients with the GG genotype. They also may have decreased clearance of isoniazid as compared to those with the AG or GG genotype. Other genetic and clinical factors may also influence risk for hepatotoxicity and clearance of isoniazid.\item \textbf{\colorbox{teal} {Class 3}} \textbf{ rs1799931 } \textit{ GG }
\item[] Patients with the GG genotype and tuberculosis (TB) may have a decreased risk of hepatotoxicity when treated with anti-TB drugs as compared to patients with the AA or AG genotype. However, some studies find no association with hepatotoxicity. Other genetic and clinical factors may also influence risk of hepatotoxicity.
\end{rSubsection}

\end{rSection}\begin{rSection}{ risperidone }
\item[]

\begin{rSubsection}{ ABCB1 }{ ATP-binding cassette, sub-family B (MDR/TAP), member 1 }{}{}
\item[]

\item[] ---------------------------------------------------- Clinical Annotations -----------------------------------------------------\newline
\item \textbf{\colorbox{teal} {Class 3}} \textbf{ rs1045642 } \textit{ GG }
\item[] Patients with the GG genotype and schizophrenia may have a shorter QTc interval when treated with risperidone as compared to patients with the AA or AG genotype. Other genetic and clinical factors may also influence QTc interval in patients taking risperidone.
\end{rSubsection}

\end{rSection}\begin{rSection}{ ritonavir }
\item[]

\begin{rSubsection}{ UGT1A }{ UDP glucuronosyltransferase 1 family, polypeptide A complex locus }{}{}
\item[]

\item[] ---------------------------------------------------- Clinical Annotations -----------------------------------------------------\newline
\item \textbf{\colorbox{teal} {Class 3}} \textbf{ rs10929303 } \textit{ TC }
\item[] Patients with the CT genotype and HIV may have a decreased risk of nephrolithiasis when treated with atazanavir and ritonavir as compared to patients with the TT genotype and an increased risk of nephrolithiasis as compared to people with the CC genotype. Other genetic and clinical factors may also affect risk of nephrolithiasis in patients with HIV who are taking atazanavir and ritonavir. \item \textbf{\colorbox{teal} {Class 3}} \textbf{ rs1042640 } \textit{ GC }
\item[] Patients with the CG genotype and HIV may have a decreased risk of nephrolithiasis when treated with atazanavir and ritonavir as compared to patients with the GG genotype and an increased risk of nephrolithiasis as compared to patients with the CC genotype. Other genetic and clinical factors may also affect risk of nephrolithiasis in people with HIV who are taking atazanavir and ritonavir.\item \textbf{\colorbox{teal} {Class 3}} \textbf{ rs8330 } \textit{ GC }
\item[] Patients with the CG genotype and HIV may have a decreased risk of nephrolithiasis when treated with atazanavir and ritonavir as compared to patients with the GG genotype and an increased risk of nephrolithiasis as compared to people with the CC genotype. Other genetic and clinical factors may also affect risk of nephrolithiasis in patients with HIV who are taking atazanavir and ritonavir.
\end{rSubsection}

\end{rSection}\begin{rSection}{ rosiglitazone }
\item[]

\begin{rSubsection}{ CYP2C8 }{ cytochrome P450, family 2, subfamily C, polypeptide 8 }{}{}
\item[]

\item[] ---------------------------------------------------- Clinical Annotations -----------------------------------------------------\newline
\item \textbf{\colorbox{cyan} {Class 2A}} \textbf{ rs10509681 } \textit{ TT }
\item[] Patients with the TT (CYP2C8*1/*1) genotype may have decreased metabolism of rosiglitazone, a larger change in HbA1c, and an increased risk of edema as compared to patients with the CC (CYP2C8*3/*3) or CT (CYP2C8*3/*1) genotype. One study found no association with blood glucose levels. Other genetic and clinical factors may also influence metabolism of rosiglitazone, risk of edema and blood glucose levels.
\end{rSubsection}

\end{rSection}\begin{rSection}{ sildenafil }
\item[]

\begin{rSubsection}{ VEGFA }{ vascular endothelial growth factor A }{}{}
\item[]

\item[] ---------------------------------------------------- Clinical Annotations -----------------------------------------------------\newline
\item \textbf{\colorbox{teal} {Class 3}} \textbf{ rs699947 } \textit{ AC }
\item[] Patients with the AC genotype may have decreased response to sildenafil in men with Erectile Dysfunction as compared to patients with genotype CC. Other genetic and clinical factors may also influence the response to sildenafil.
\end{rSubsection}

\end{rSection}\begin{rSection}{ simvastatin }
\item[]

\begin{rSubsection}{ HMGCR }{ 3-hydroxy-3-methylglutaryl-CoA reductase }{}{}
\item[]

\item[] ---------------------------------------------------- Clinical Annotations -----------------------------------------------------\newline
\item \textbf{\colorbox{green} {Class 4}} \textbf{ rs3846662 } \textit{ GG }
\item[] The GG genotype may be associated with decreased induction of full-length transcripts and increased expression of spliced HMGCRv1 transcript  as compared to AA genotype.
\end{rSubsection}

\end{rSection}\begin{rSection}{ sorafenib }
\item[]

\begin{rSubsection}{ VEGFA }{ vascular endothelial growth factor A }{}{}
\item[]

\item[] ---------------------------------------------------- Clinical Annotations -----------------------------------------------------\newline
\item \textbf{\colorbox{teal} {Class 3}} \textbf{ rs2010963 } \textit{ CG }
\item[] Patients with the CG genotype may have increased risk of hand-foot syndrome when treated with sorafenib in people with Carcinoma, Renal Cell as compared to patients with genotype CC. Other genetic and clinical factors may also influence the toxicity to sorafenib.\item \textbf{\colorbox{teal} {Class 3}} \textbf{ rs1570360 } \textit{ AG }
\item[] Patients with the AG genotype may have unfavorable progression-free survival when treated with sorafenib in people with Carcinoma, Renal Cell as compared to patients with genotype GG. Other genetic and clinical factors may also influence the response to sorafenib.\item \textbf{\colorbox{teal} {Class 3}} \textbf{ rs2010963 } \textit{ CG }
\item[] Patients with the CG genotype may have increased progression-free survival and increased overall survival when treated with sorafenib in people with Hepatocellular Carcinoma as compared to patients with genotype GG. Other genetic and clinical factors may also influence the response to sorafenib.
\end{rSubsection}

\end{rSection}\begin{rSection}{ sulfonamides, urea derivatives }
\item[]

\begin{rSubsection}{ CYP2C9 }{ cytochrome P450, family 2, subfamily C, polypeptide 9 }{}{}
\item[]

\item[] ---------------------------------------------------- Clinical Annotations -----------------------------------------------------\newline
\item \textbf{\colorbox{teal} {Class 3}} \textbf{ rs1057910 } \textit{ AC }
\item[] Results from patients with the AC genotype were not statistically significant.
\end{rSubsection}

\end{rSection}\begin{rSection}{ sunitinib }
\item[]

\begin{rSubsection}{ VEGFA }{ vascular endothelial growth factor A }{}{}
\item[]

\item[] ---------------------------------------------------- Clinical Annotations -----------------------------------------------------\newline
\item \textbf{\colorbox{teal} {Class 3}} \textbf{ rs699947 } \textit{ AC }
\item[] Patients with the AC genotype may have higher increase in systolic blood pressure and increased risk of developing grade 3 hypertension when treated with sunitinib as compared to patients with genotype CC. Other genetic and clinical factors may also influence the response to sunitinib.
\end{rSubsection}

\end{rSection}\begin{rSection}{ tacrolimus }
\item[]

\begin{rSubsection}{ ABCB1 }{ ATP-binding cassette, sub-family B (MDR/TAP), member 1 }{}{}
\item[]

\item[] ---------------------------------------------------- Clinical Annotations -----------------------------------------------------\newline
\item \textbf{\colorbox{teal} {Class 3}} \textbf{ rs1045642 } \textit{ GG }
\item[] Patients with the GG genotype who are undergoing organ transplantation may have increased clearance and dose requirements of tacrolimus, as compared to patients with the AA or AG genotype. However, the vast majority of studies find no association between this SNP and clearance or dose of tacrolimus. Other genetic and clinical factors, such as CYP3A5*3, may also influence clearance and dose of tacrolimus. \item \textbf{\colorbox{teal} {Class 3}} \textbf{ rs1045642 } \textit{ GG }
\item[] Patients with the GG genotype who are CYP2C19 extensive metabolizers and are receiving tacrolimus after renal transplantation may have increased plasma concentrations of (R)-lansoprazole but no significant differences in the frequency of gastroesophageal symptoms as compared to patients with the AA genotype. Other genetic and clinical factors may also influence lansoprazole clearance.\item \textbf{\colorbox{teal} {Class 3}} \textbf{ rs1045642 } \textit{ GG }
\item[] Pediatric patients with the GG genotype who are treated with prednisone and tacrolimus may have an increased risk of remaining on steroids 1 year after heart transplantation compared to patients with the AA or AG genotype. Other genetic and clinical factors may also influence risk of remaining on steroids 1 year after transplantation.\item \textbf{\colorbox{teal} {Class 3}} \textbf{ rs1045642 } \textit{ GG }
\item[] Patients who receive a kidney with the GG genotype may have increased estimated glomerular filtration rate (eGFR) when treated with tacrolimus as compared to patients with the AA or AG genotype. No significant results were seen when recipient genotype was considered. Other genetic and clinical factors may also influence eGFR. \item \textbf{\colorbox{teal} {Class 3}} \textbf{ rs2032582 } \textit{ CC }
\item[] Patients with CC genotype may have lower success rate in achieving short-term remission when treated with tacrolimus in people with Colitis, Ulcerative as compared to patients with the AA genotype. The majority of studies find no association with dose of tacrolimus in people with transplantations as compared and genotypes of this SNP. Other genetic or clinical factors may influence response and dose of tacrolimus.\item \textbf{\colorbox{teal} {Class 3}} \textbf{ rs1128503 } \textit{ GG }
\item[] Patients with the GG genotype who are undergoing organ transplantation may have decreased concentrations of tacrolimus as compared to patients with the AA or AG genotype. However, the majority of the literature evidence shows no association between this variant and tacrolimus concentrations, clearance or dose. Other genetic and clinical factors may also influence concentrations of tacrolimus. \item \textbf{\colorbox{teal} {Class 3}} \textbf{ rs2032582 } \textit{ CC }
\item[] Patients with the CC genotype who are undergoing organ transplantation may have increased metabolism and dose requirements of tacrolimus, as compared to patients with the AA, AC, CT or TT genotypes. However, the majority of studies have found no association between this polymorphism and metabolism or dose of tacrolimus. Other genetic and clinical factors, such as CYP3A5*3, may also influence metabolism and dose of tacrolimus.\item \textbf{\colorbox{teal} {Class 3}} \textbf{ rs1045642 } \textit{ GG }
\item[] Patients with the GG genotype and ulcerative colitis may have a poorer chance at achieving remission when treated with tacrolimus as compared to patients with the AA genotype. Other genetic and clinical factors may also influence likelihood of ulcerative colitis remission.\item \textbf{\colorbox{teal} {Class 3}} \textbf{ rs1045642 } \textit{ GG }
\item[] Patients with the GG genotype who are undergoing kidney transplantation and are treated with tacrolimus may have decreased risk of experiencing transplant rejection as compared to patients with the AG genotype. However, the majority of studies find no association between this polymorphism and risk for transplant rejection. Other genetic and clinical factors may also influence risk of transplant rejection.\item \textbf{\colorbox{teal} {Class 3}} \textbf{ rs1045642 } \textit{ GG }
\item[] Patients with the GG genotype who are undergoing kidney transplantation may have a decreased risk of hypokalemia when treated with tacrolimus as compared to patients with the AG genotype. Other genetic and clinical factors may also influence risk of hypokalemia.
\end{rSubsection}

\end{rSection}\begin{rSection}{ tamoxifen }
\item[]

\begin{rSubsection}{ ABCB1 }{ ATP-binding cassette, sub-family B (MDR/TAP), member 1 }{}{}
\item[]

\item[] ---------------------------------------------------- Clinical Annotations -----------------------------------------------------\newline
\item \textbf{\colorbox{teal} {Class 3}} \textbf{ rs1045642 } \textit{ GG }
\item[] Women with the GG genotype and breast cancer may have a decreased chance of disease recurrence when treated with tamoxifen as compared to patients with the AG genotype. Other genetic and clinical factors may also influence breast cancer recurrence.
\end{rSubsection}

\end{rSection}\begin{rSection}{ tegafur }
\item[]

\begin{rSubsection}{ DPYD }{ dihydropyrimidine dehydrogenase }{}{}
\item[]
\item[] ------------------------------------------------------ Dosing Guideline --------------------------------------------------------\newline
\item[]
\item[] \textbf{ DPYD:*1/*1 } Moderate

\item[] Use label-recommended dosage and administration.
\item[] ---------------------------------------------------- Clinical Annotations -----------------------------------------------------\newline
\item \textbf{\colorbox{red} {Class 1A}} \textbf{ rs55886062 } \textit{ AA }
\item[] Patients with the AA genotype (DPYD *1/*1) and cancer who are treated with fluoropyrimidine-based chemotherapy may have a decreased, but not absent, risk for drug toxicity as compared to patients with the AC or CC genotype (DPYD *1/*13 or *13/*13). Fluoropyrimidines are often used in combination chemotherapy such as FOLFOX (fluorouracil, leucovorin and oxaliplatin), FOLFIRI (fluorouracil, leucovorin and irinotecan) or FEC (fluorouracil, epirubicin and cyclophosphamide) or with other drugs such as bevacizumab, cetuximab, raltitrexed. The combination and delivery of the drug may influence risk for toxicity. Other genetic and clinical factors may also influence response to fluoropyrimidine-based chemotherapy.\item \textbf{\colorbox{red} {Class 1A}} \textbf{ rs3918290 } \textit{ CC }
\item[] Patients with the CC genotype (DPYD *1/*1) and cancer who are treated with fluoropyrimidine-based chemotherapy may have 1) increased clearance of fluoropyrimidine drugs and 2) decreased, but not non-existent, risk for drug toxicity as compared to patients with the CT or TT genotype (DPYD *1/*2A or *2A/*2A). Fluoropyrimidines are often used in combination chemotherapy such as FOLFOX (fluorouracil, leucovorin and oxaliplatin), FOLFIRI (fluorouracil,  leucovorin and irinotecan) or FEC (fluorouracil, epirubicin and cyclophosphamide) or with other drugs such as bevacizumab, cetuximab, raltitrexed. The combination and delivery of the drug may influence risk for toxicity. Other genetic and clinical factors may also influence response to fluoropyrimidine based chemotherapy.\item \textbf{\colorbox{red} {Class 1A}} \textbf{ rs67376798 } \textit{ TT }
\item[] Patients with the TT genotype and cancer who are treated with fluoropyrimidine-based chemotherapy may have 1) increased clearance of the drug and 2) decreased, but not absent, risk and reduced severity of drug toxicity as compared to patients with the AT genotype. Fluoropyrimidines are often used in combination chemotherapy such as FOLFOX (fluorouracil, leucovorin and oxaliplatin), FOLFIRI (fluorouracil, leucovorin and irinotecan) or FEC (fluorouracil, epirubicin and cyclophosphamide) or with other drugs such as bevacizumab, cetuximab, raltitrexed. The combination and delivery of the drug may influence risk for toxicity. Other genetic and clinical factors may also influence response to fluoropyrimidine-based chemotherapy.\item \textbf{\colorbox{teal} {Class 3}} \textbf{ rs1801159 } \textit{ TT }
\item[] Patients with the TT genotype (DPYD *1/*1) and cancer who are treated with fluoropyrimidine-based chemotherapy may have 1) a decreased likelihood of nausea, vomiting, and leukopenia, 2) increased response and 3) increased clearance of fluorouracil as compared to patients with the CT or CC genotype (DPYD *1/*5 or *5/*5). However, other studies find no associations or contradictory associations with fluoropyrimidine-induced drug toxicity or response. Other genetic and clinical factors may also influence response to fluoropyrimidine-based chemotherapy.
\end{rSubsection}

\end{rSection}\begin{rSection}{ thalidomide }
\item[]

\begin{rSubsection}{ CYP4B1 }{ cytochrome P450, family 4, subfamily B, polypeptide 1 }{}{}
\item[]

\item[] ---------------------------------------------------- Clinical Annotations -----------------------------------------------------\newline
\item \textbf{\colorbox{teal} {Class 3}} \textbf{ rs4646487 } \textit{ CC }
\item[] Patients with the CC genotype may have a decreased but not absent risk of toxicity with docetaxel and thalidomide as compared to patients with the CT or TT genotypes. Other genetic and clinical factors may also influence treatment response.
\end{rSubsection}

\end{rSection}\begin{rSection}{ ticagrelor }
\item[]

\begin{rSubsection}{ CYP3A4 }{ cytochrome P450, family 3, subfamily A, polypeptide 4 }{}{}
\item[]

\item[] ---------------------------------------------------- Clinical Annotations -----------------------------------------------------\newline
\item \textbf{\colorbox{teal} {Class 3}} \textbf{ rs56324128 } \textit{ CC }
\item[] Patients with the CC genotype and acute coronary syndrome may have decreased concentrations of ticagrelor compared to patients with the CT genotype. Other factors may affect concentrations of ticagrelor.
\end{rSubsection}

\end{rSection}\begin{rSection}{ tolbutamide }
\item[]

\begin{rSubsection}{ CYP2C9 }{ cytochrome P450, family 2, subfamily C, polypeptide 9 }{}{}
\item[]

\item[] ---------------------------------------------------- Clinical Annotations -----------------------------------------------------\newline
\item \textbf{\colorbox{green} {Class 4}} \textbf{ rs9332239 } \textit{ CC }
\item[] Patients with the CC genotype may have increased metabolism of tolbutamide as compared to patients with the CT or TT genotypes. Other genetic and clinical factors may also influence tolbutamide metabolism.
\end{rSubsection}

\end{rSection}\begin{rSection}{ tramadol }
\item[]

\begin{rSubsection}{ SLC22A1 }{ solute carrier family 22 (organic cation transporter), member 1 }{}{}
\item[]

\item[] ---------------------------------------------------- Clinical Annotations -----------------------------------------------------\newline
\item \textbf{\colorbox{teal} {Class 3}} \textbf{ rs12208357 } \textit{ CC }
\item[] Patients with the CC genotype may have lower plasma concentrations of O-desmethyltramadol when exposed to tramadol in healthy individuals as compared to patients with the TT genotype. Other genetic or clinical factors may influence the response to tramadol.\item \textbf{\colorbox{teal} {Class 3}} \textbf{ rs34130495 } \textit{ GG }
\item[] Patients with the GG genotype may have decreased plasma concentrations of O-desmethyltramadol when exposed to tramadol in healthy individuals as compared to patients with the AA or AG genotype. Other genetic or clinical factors may also influence the clearance of tramadol.
\end{rSubsection}

\end{rSection}\begin{rSection}{ valproic acid }
\item[]

\begin{rSubsection}{ UGT1A10 }{ UDP glucuronosyltransferase 1 family, polypeptide A10 }{}{}
\item[]

\item[] ---------------------------------------------------- Clinical Annotations -----------------------------------------------------\newline
\item \textbf{\colorbox{teal} {Class 3}} \textbf{ rs2070959 } \textit{ AG }
\item[] Patients with the AG genotype may require an increased dose of valproic acid compared to patients with the AA genotype. Other genetic and clinical factors may also influence a patients dose requirements.
\end{rSubsection}

\end{rSection}\begin{rSection}{ venlafaxine }
\item[]

\begin{rSubsection}{ ABCB1 }{ ATP-binding cassette, sub-family B (MDR/TAP), member 1 }{}{}
\item[]

\item[] ---------------------------------------------------- Clinical Annotations -----------------------------------------------------\newline
\item \textbf{\colorbox{teal} {Class 3}} \textbf{ rs1045642 } \textit{ GG }
\item[] Patients with genotype GG and depressive disorder may have increased response to venlafaxine compared to patients with genotype AA or AG. Patients with GG genotype and narcolepsy were not found to have different response to venlafaxine compared to patients with other genotypes. Other clinical and genetic factors also may affect response to venlafaxine.
\end{rSubsection}

\end{rSection}\begin{rSection}{ verapamil }
\item[]

\begin{rSubsection}{ ABCB1 }{ ATP-binding cassette, sub-family B (MDR/TAP), member 1 }{}{}
\item[]

\item[] ---------------------------------------------------- Clinical Annotations -----------------------------------------------------\newline
\item \textbf{\colorbox{teal} {Class 3}} \textbf{ rs2032582 } \textit{ CC }
\item[] Patients with the CC genotype may have decreased metabolism of verapamil as compared to patients with the AA or AC genotype. Other genetic and clinical factors may also impact the metabolism of verapamil.\item \textbf{\colorbox{teal} {Class 3}} \textbf{ rs1045642 } \textit{ GG }
\item[] Patients with the GG genotype may have decreased metabolism of verapamil as compared to patients with the AA or AG genotype. Other genetic and clinical factors may also impact the metabolism of verapamil.
\end{rSubsection}

\end{rSection}\begin{rSection}{ vitamin e }
\item[]

\begin{rSubsection}{ CYP4F2 }{ cytochrome P450, family 4, subfamily F, polypeptide 2 }{}{}
\item[]

\item[] ---------------------------------------------------- Clinical Annotations -----------------------------------------------------\newline
\item \textbf{\colorbox{teal} {Class 3}} \textbf{ rs2108622 } \textit{ CC }
\item[] Patients with the CC genotype may have decreased steady-state levels of vitamin E when taking vitamin E supplements as compared to patients with the CT and TT genotypes. Other clinical and genetic factors may also influence steady-state levels of vitamin E in patients taking vitamin E supplements.\item \textbf{\colorbox{green} {Class 4}} \textbf{ rs3093105 } \textit{ AA }
\item[] The AA genotype may be associated with decreased CYP4F2 activity and decreased vitamin e metabolism as compared to the AC or CC genotype. This is based solely on an in vitro study in a haploid heterologous cell system. Other clinical and genetic factors may also influence metabolism of vitamin e.
\end{rSubsection}

\end{rSection}\begin{rSection}{ warfarin }
\item[]

\begin{rSubsection}{ CYP2C9 }{ cytochrome P450, family 2, subfamily C, polypeptide 9 }{}{}
\item[]
\item[] ------------------------------------------------------ Dosing Guideline --------------------------------------------------------\newline
\item[]
\item[] \textbf{ CYP2C9:*1/*1 } N/A

\item[] Estimate the anticipated stable dose of warfarin using the algorithms available on http://www.warfarindosing.org, the IWPC Pharmacogenetic Dosing Algorithm or the FDA-approved drug label
\item[] ---------------------------------------------------- Clinical Annotations -----------------------------------------------------\newline
\item \textbf{\colorbox{red} {Class 1A}} \textbf{ rs1057910 } \textit{ AC }
\item[] Patients with the AC genotype: 1) may require a decreased dose of warfarin as compared to patients with the AA genotype 2) may have an increased risk for adverse events as compared to patients with the AA genotype.\item \textbf{\colorbox{cyan} {Class 2A}} \textbf{ rs7900194 } \textit{ GG }
\item[] Patients with the GG genotype who are treated with warfarin may require a higher maintenance dose as compared to patients with the AG or GG genotype.  Other clinical or genetic factors may also influence warfarin dose.\item \textbf{\colorbox{cyan} {Class 2A}} \textbf{ rs56165452 } \textit{ TT }
\item[] Patients with the TT genotype may required higher dose of warfarin as compared to patients with the CT or CC genotype. Other clinical or genetic factors may also influence  warfarin dose. This variant rs56165452 defines CYP2C9*4.
\end{rSubsection}

\end{rSection}\begin{rSection}{ zidovudine }
\item[]

\begin{rSubsection}{ ABCB1 }{ ATP-binding cassette, sub-family B (MDR/TAP), member 1 }{}{}
\item[]

\item[] ---------------------------------------------------- Clinical Annotations -----------------------------------------------------\newline
\item \textbf{\colorbox{teal} {Class 3}} \textbf{ rs1045642 } \textit{ GG }
\item[] Patients with the GG genotype and HIV may have an increased risk of virological failure when receiving highly active antiretroviral therapy (HAART), as compared to patients with the AA genotype. Other genetic and clinical factors may also influence risk of virological failure on HAART.
\end{rSubsection}

\end{rSection}

\end{document}