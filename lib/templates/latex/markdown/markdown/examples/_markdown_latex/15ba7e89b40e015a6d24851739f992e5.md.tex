The Royal Dutch Pharmacists Association - Pharmacogenetics Working Group has evaluated therapeutic dose recommendations for atomoxetine based on CYP2D6 genotypes [PMID:21412232].\markdownRendererInterblockSeparator
{}\markdownRendererPipe{} Phenotype (Genotype) \markdownRendererPipe{} Therapeutic Dose Recommendation \markdownRendererPipe{} Level of Evidence \markdownRendererPipe{} Clinical Relevance \markdownRendererPipe{} \markdownRendererPipe{} --- \markdownRendererPipe{} --- \markdownRendererPipe{} --- \markdownRendererPipe{} --- \markdownRendererPipe{} \markdownRendererPipe{} PM (two inactive (\markdownRendererEmphasis{3-}8, \markdownRendererEmphasis{11-}16, \markdownRendererEmphasis{19-}21, *38, \markdownRendererEmphasis{40, \markdownRendererEmphasis{42) alleles) \markdownRendererPipe{} Standard dose. Dose increase probably not necessary; be alert to ADEs. \markdownRendererPipe{} Published controlled studies of moderate quality} relating to phenotyped and/or genotyped patients or healthy volunteers, and having relevant pharmacokinetic or clinical endpoints. \markdownRendererPipe{} Clinical effect (S): short-lived discomfort (\markdownRendererAmpersand{}lt; 48 hr) without permanent injury: e.g. reduced decrease in resting heart rate; reduction in exercise tachycardia; decreased pain relief from oxycodone; ADE resulting from increased bioavailability of atomoxetine (decreased appetite, insomnia, sleep disturbance etc); neutropenia \markdownRendererAmpersand{}gt; 1.5x10\markdownRendererCircumflex{}9\markdownRendererCircumflex{}/l; leucopenia \markdownRendererAmpersand{}gt; 3.0x10\markdownRendererCircumflex{}9\markdownRendererCircumflex{}/l; thrombocytopenia \markdownRendererAmpersand{}gt; 75x10\markdownRendererCircumflex{}9\markdownRendererCircumflex{}/l; moderate diarrhea not affecting daily activities; reduced glucose increase following oral glucose tolerance test. \markdownRendererPipe{} \markdownRendererPipe{} IM (two decreased-activity (}9, *10, *17, *29, *36, \markdownRendererEmphasis{41) alleles or carrying one active (}1, *2, \markdownRendererEmphasis{33, \markdownRendererEmphasis{35) and one inactive (}3-}8, \markdownRendererEmphasis{11-}16, \markdownRendererEmphasis{19-}21, *38, *40, \markdownRendererEmphasis{42) allele, or carrying one decreased-activity (}9, *10, *17, *29, \markdownRendererEmphasis{36, \markdownRendererEmphasis{41) allele and one inactive (}3-}8, \markdownRendererEmphasis{11-}16, \markdownRendererEmphasis{19-}21, \markdownRendererEmphasis{38, \markdownRendererEmphasis{40, \markdownRendererEmphasis{42) allele) \markdownRendererPipe{} No recommendations. \markdownRendererPipe{} Published controlled studies of good quality} relating to phenotyped and/or genotyped patients or healthy volunteers, and having relevant pharmacokinetic or clinical endpoints. \markdownRendererPipe{}Minor clinical effect (S): QTc prolongation (\markdownRendererAmpersand{}lt;450 ms female, \markdownRendererAmpersand{}lt;470 ms male); INR increase \markdownRendererAmpersand{}lt; 4.5; Kinetic effect (S). \markdownRendererPipe{} \markdownRendererPipe{} UM (a gene duplication in absence of inactive (}3-}8, \markdownRendererEmphasis{11-}16, \markdownRendererEmphasis{19-}21, *38, *40, \markdownRendererEmphasis{42) or decreased-activity (}9, *10, *17, *29, *36, *41) alleles) \markdownRendererPipe{} Insufficient data to allow calculation of dose adjustment. Be alert to reduced efficacy or select alternative drug (e.g., methylphenidate, clonidine). \markdownRendererPipe{} -- \markdownRendererPipe{} -- \markdownRendererPipe{}\markdownRendererInterblockSeparator
{}\markdownRendererUlBeginTight
\markdownRendererUlItem *See [Methods]( http://www.pharmgkb.org/home/dutch\markdownRendererEmphasis{pharmacogenetics}working\markdownRendererUnderscore{}group.jsp) or [PMID: 18253145] for definition of \markdownRendererAmpersand{}\markdownRendererHash{}34;good\markdownRendererAmpersand{}\markdownRendererHash{}34; and \markdownRendererAmpersand{}\markdownRendererHash{}34;moderate\markdownRendererAmpersand{}\markdownRendererHash{}34; quality.\markdownRendererUlItemEnd 
\markdownRendererUlItem S: statistically significant difference.\markdownRendererUlItemEnd 
\markdownRendererUlItem Please see attached PDF for detailed information about the evaluated studies: \markdownRendererLink{Atomoxetine CYP2D6}{\markdownRendererCircumflex{}atomoxetine\markdownRendererUnderscore{}CYP2D6\markdownRendererUnderscore{}271111.pdf}{^atomoxetine_CYP2D6_271111.pdf}{}\markdownRendererUlItemEnd 
\markdownRendererUlEndTight \relax