\documentclass{book}
\usepackage{ifxetex,ifluatex}
\ifxetex
  \usepackage{polyglossia}
  \setmainlanguage{english}
  \usepackage{fontspec}
\else\ifluatex
  \usepackage{polyglossia}
  \setmainlanguage{english}
  \usepackage{fontspec}
\else
  \usepackage[english]{babel}
  \usepackage[utf8]{inputenc}
  \usepackage[T1]{fontenc}
  \usepackage{lmodern}
\fi\fi
\usepackage[
  hashEnumerators,
  definitionLists,
  footnotes,
  inlineFootnotes,
  smartEllipses,
  fencedCode,
]{markdown}
\begin{document}

  \begin{markdown}
The Royal Dutch Pharmacists Association - Pharmacogenetics Working Group has evaluated therapeutic dose recommendations for atomoxetine based on CYP2D6 genotypes [PMID:21412232].  \newline

| Phenotype (Genotype) | Therapeutic Dose Recommendation | Level of Evidence | Clinical Relevance |
| --- | --- | --- | --- |
| PM (two inactive (*3-*8, *11-*16, *19-*21, *38, *40, *42) alleles) | Standard dose. Dose increase probably not necessary; be alert to ADEs. | Published controlled studies of moderate quality* relating to phenotyped and/or genotyped patients or healthy volunteers, and having relevant pharmacokinetic or clinical endpoints. | Clinical effect (S): short-lived discomfort (&lt; 48 hr) without permanent injury: e.g. reduced decrease in resting heart rate; reduction in exercise tachycardia; decreased pain relief from oxycodone; ADE resulting from increased bioavailability of atomoxetine (decreased appetite, insomnia, sleep disturbance etc); neutropenia &gt; 1.5x10^9^/l; leucopenia &gt; 3.0x10^9^/l; thrombocytopenia  &gt; 75x10^9^/l; moderate diarrhea not affecting daily activities; reduced glucose increase following oral glucose tolerance test. |
| IM (two decreased-activity (*9, *10, *17, *29, *36, *41) alleles or carrying one active (*1, *2, *33, *35) and one inactive (*3-*8, *11-*16, *19-*21, *38, *40, *42) allele, or carrying one decreased-activity (*9, *10, *17, *29, *36, *41) allele and one inactive (*3-*8, *11-*16, *19-*21, *38, *40, *42) allele) | No recommendations. | Published controlled studies of good quality* relating to phenotyped and/or genotyped patients or healthy volunteers, and having relevant pharmacokinetic or clinical endpoints. |Minor clinical effect (S): QTc prolongation (&lt;450 ms female, &lt;470 ms male); INR increase &lt; 4.5; Kinetic effect (S). | 
| UM (a gene duplication in absence of inactive (*3-*8, *11-*16, *19-*21, *38, *40, *42) or decreased-activity (*9, *10, *17, *29, *36, *41) alleles) | Insufficient data to allow calculation of dose adjustment. Be alert to reduced efficacy or select alternative drug (e.g., methylphenidate, clonidine). | -- | -- | 

- *See [Methods ]( http://www.pharmgkb.org/home/dutch_pharmacogenetics_working_group.jsp) or [PMID: 18253145] for definition of &#34;good&#34; and &#34;moderate&#34; quality.
- S: statistically significant difference.
- Please see attached PDF for detailed information about the evaluated studies: [Atomoxetine CYP2D6 ](^atomoxetine_CYP2D6_271111.pdf)
  \end{markdown}
\end{document}
