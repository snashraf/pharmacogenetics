\begin{rSubsection}{ ADRB2 }{ adrenoceptor beta 2, surface }{}{}
\item[] ------------------------------------------------------ Dosing Guideline --------------------------------------------------------\newline
\item The Royal Dutch Pharmacists Association - Pharmacogenetics Working Group has evaluated therapeutic dose recommendations for duloxetine based on CYP2D6 genotypes [PMID:21412232].  They conclude that there are no recommendations at this time.
\vspace{1pt}\\
\scriptsize
\begin{center}
\begin{tabularx}{0.9\textwidth}{ bsss }
\textbf{Phenotype (Genotype)} & \textbf{Therapeutic Dose Recommendation} & \textbf{Level of Evidence} & \textbf{Clinical Relevance} \\
\vspace{1pt}\\
\hline \\
\vspace{1pt}\\
PM (two inactive (*3-*8, *11-*16, *19-*21, *38, *40, *42) alleles) & No recommendation. & Data on file. & Clinical effect (not statistically significant difference). \\
\vspace{1pt}\\
\hline \\
\vspace{1pt}\\
IM (two decreased-activity (*9, *10, *17, *29, *36, *41) alleles or carrying one active (*1, *2, *33, *35) and one inactive (*3-*8, *11-*16, *19-*21, *38, *40, *42) allele, or carrying one decreased-activity (*9, *10, *17, *29, *36, *41) allele and one inactive (*3-*8, *11-*16, *19-*21, *38, *40, *42) allele) & No recommendation. & No evidence. & --  \\
\vspace{1pt}\\
\hline \\
\vspace{1pt}\\
UM (a gene duplication in absence of inactive (*3-*8, *11-*16, *19-*21, *38, *40, *42) or decreased-activity (*9, *10, *17, *29, *36, *41) alleles) & No recommendation. & No evidence. & -- \\
\end{tabularx}
\end{center}
\normalsize
\vspace{10pt}
